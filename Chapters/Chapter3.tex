\chapter{Diseño e implementación} % Main chapter title

\label{Chapter3}

En este capítulo se describe la arquitectura global del prototipo, se detalla cada módulo de hardware y software que lo compone, y se documentan las decisiones de implementación y los criterios de diseño. Se explican los flujos de datos entre el dispositivo de campo, el broker MQTT, el backend (API REST), la interfaz web, se resumen las consideraciones para el despliegue y el monitoreo post-implantación.


\section{Arquitectura del sistema}

La arquitectura propuesta separa de forma explícita el dispositivo de campo (contador + ESP32-C3 + SIM800L), el transporte de mensajes (broker MQTT) y los servicios de aplicación (API REST, persistencia y frontend). Esta separación facilita la interoperabilidad y permite desplegar la solución de forma local, remota o híbrida según las políticas institucionales

El sistema se organiza en cinco bloques con funciones definidas que aseguran un flujo de datos confiable, eficiente y seguro durante la captura, transmisión, procesamiento y visualización de eventos, garantizando trazabilidad, persistencia y control remoto.

\begin{itemize}

\item {Dispositivo de campo:} integra el contador (RS-232), un ESP32-C3 y un módem SIM800L. El firmware, basado en \texttt{ESP-IDF}, lee y parsea tramas, valida y normaliza campos, agrega sello UTC, encola eventos y publica por MQTT. Gestiona reintentos, comandos y telemetría, y mantiene una persistencia mínima (últimas tramas y comandos pendientes) para recuperación tras reinicio.

\item {Transporte (broker MQTT):} funciona como bus de mensajes desacoplado. Se sugiere usar Eclipse Mosquitto en la etapa inicial y considerar brokers gestionados para escalar. Implementa autenticación, control de tópicos y cifrado. Se emplean tópicos jerárquicos por dispositivo para facilitar filtrado y autorización: 
\begin{itemize}
  \item \texttt{dispositivo/\{id\}/medicion}
  \item \texttt{dispositivo/\{id\}/comando} 
  \item \texttt{dispositivo/\{id\}/respuesta}
\end{itemize}


\item {Servidor central:} el servidor central reúne dos responsabilidades principales: 

\begin{itemize}  

\item  Componente suscriptor MQTT que valida, transforma y enruta mensajes hacia la lógica de negocio y la persistencia.

\item API REST  ofrece servicios de consulta, gestión y comandos, manteniendo independencia del broker para permitir nuevos consumidores. Los datos se almacenan en MySQL con soporte para rangos temporales, alto rendimiento y auditoría de comandos.
 \end{itemize}

\item Cliente/Visualización: la interfaz web, desarrollada en Ionic + Angular, consume la API REST para consultas históricas y eventos en tiempo real. Ofrece visualización de eventos, consultas filtradas, envío de comandos y un panel de telemetría para mantenimiento.
\end{itemize}

Se separó el broker de la aplicación, se usó MQTT por su eficiencia y se creó una API REST en Node.js para gestión y autenticación.

La figura \ref{fig:diag_arquitectura} muestra el diagrama de arquitectura del sistema y el flujo de datos.


\begin{figure}[H]
  \centering
  \includegraphics[width=0.45\linewidth]{./Figures/diagArq.png}
  \caption{Diagrama de arquitectura del sistema y el flujo de datos.}
  \label{fig:diag_arquitectura}
\end{figure}



\subsection{Flujo de datos} 
El flujo de datos del sistema describe cómo se captura, procesa y comunica la información desde el contador hasta la interfaz de usuario, lo que permite la trazabilidad, persistencia y control de los eventos y comandos. Se detallan las etapas principales:

\begin{itemize}

  \item Detección: el contador detecta un paso, acumula y en determinado intervalo envía una trama por RS-232 al ESP32-C3.

  \item Preprocesado en nodo: el firmware valida la trama, añade sello temporal y metadatos, y encola el evento en memoria (FIFO).

  \item Transmisión: cuando la conexión GPRS está disponible, el nodo publica las mediciones en el tópico MQTT \texttt{dispositivo/{id}/medicion}.
  
  \item Ingesta y persistencia: el broker Mosquitto entrega el mensaje al suscriptor backend, el servicio valida el payload y persiste el registro en la base de datos MySQL.
  
  \item Visualización/Control: la interfaz web consulta la API REST para datos históricos y recibe notificaciones en tiempo real.
  
  \item Emisión de comandos (desde UI): el operador genera un comando en la interfaz, la UI envía
  \texttt{POST /app/comando} al backend, que crea un \texttt{comm\_id} único y publica en \texttt{dispositivo/{id}/comando}.
  
  \item Recepción nodo/Entrega al contador: el ESP32-C3 en \\ \texttt{dispositivo/{id}/comando} recibe el comando, valida \texttt{cmd\_id} y lo envía al contador por RS-232, se aplica un timeout configurable por comando.
  
  \item Ejecución y ack: el contador ejecuta la orden y responde por RS-232, el firmware publica el ack/resultado en \texttt{dispositivo/{id}/respuesta} con \texttt{cmd\_id} y \texttt{status} (\texttt{ok}, \texttt{failed}, \texttt{timeout}, \texttt{value}).

  \item Actualización en backend y UI: el suscriptor MQTT del backend recibe el ack, actualiza la tabla \texttt{respuesta} (campo \texttt{valor}, \texttt{ack\_ts}) y notifica a la UI para que el operador vea el resultado.
\end{itemize}
\pagebreak
La figura \ref{fig:diag_secuencia} muestra el diagrama de secuencia del flujo de datos.

\begin{figure}[H]
  \centering
  \includegraphics[width=1.15\linewidth]{./Figures/diagSecuencia.png}
  \caption{Diagrama de secuencia del flujo de datos.}
  \label{fig:diag_secuencia}
\end{figure}



\section{Arquitectura del nodo}
La arquitectura de cada nodo se diseñó con el objetivo de reutilizar los contadores de tránsito actualmente desplegados en las rutas nacionales. Cada nodo integra varios módulos que permiten la adquisición, procesamiento y transmisión de los datos de tránsito:

\begin{itemize}
    \item Contador de tránsito modelo DTEC: dispositivo encargado de detectar el paso de vehículos y generar tramas de datos con la información del evento.
   
    \item Módulo de adaptación RS-232/TTL (MAX232): circuito de conversión de niveles eléctricos que asegura compatibilidad entre la interfaz serial del contador (RS-232) y el microcontrolador (niveles TTL).
      
    \item ESP32-C3: microcontrolador que ejecuta el firmware desarrollado sobre ESP-IDF. Sus funciones incluyen el preprocesamiento de eventos, el encolamiento FIFO, la suscripción a comandos remotos y la gestión integral de la comunicación con el servidor.
   
    \item Módulo de comunicación GPRS (SIM800L): interfaz de conectividad celular que publica eventos en el broker MQTT, recibe comandos desde el servidor y retransmite respuestas o estados del nodo.
    



%
%\subsection{ESP32-C3 (unidad de control)}
%El ESP32-C3 constituye el núcleo de procesamiento del nodo de campo. Se trata de un microcontrolador de bajo consumo con conectividad, elegido principalmente por su capacidad de cómputo, su soporte de entornos de desarrollo abiertos y la disponibilidad de bibliotecas optimizadas para protocolos de comunicación.
%
%\begin{itemize}
%
%\item Rol en la arquitectura: coordina la recepción de eventos desde el contador a través de RS-232, gestiona el encolamiento FIFO, controla la comunicación con el módem SIM800L mediante comandos AT y actúa como cliente MQTT.
%
%\item Entorno de desarrollo: el firmware se desarrolló sobre ESP-IDF, el framework oficial de Espressif, que permite gestionar tareas concurrentes mediante FreeRTOS y facilita la integración de librerías de red y drivers UART.
%
%\item Ventajas técnicas: bajo costo, consumo reducido, capacidad de ejecutar varias tareas en paralelo y soporte nativo para criptografía y seguridad en comunicaciones.
%
%\item Consideraciones de diseño: se provee una correcta disipación térmica y  estabilidad de la alimentación, especialmente durante la transmisión de datos.
%
%\end{itemize}
%
%\subsection{Módulo GPRS SIM800L}
%
%El módulo SIM800L implementa la conectividad celular GPRS, que permite la transmisión bidireccional de datos entre dispositivos remotos y servidor centralizado.
%
%
%\begin{itemize}
%
%\item Funciones principales: establecimiento de sesiones TCP/IP sobre GPRS, controlado por el ESP32-C3 mediante comandos AT. Soporta publicación y suscripción MQTT a través de sockets TCP persistentes.
%
%\item Integración con ESP32-C3: la comunicación entre ambos se realiza mediante UART secundaria. El firmware implementa comandos de inicialización, registro en red, apertura de contexto y gestión de reconexiones.
%
%\item Aspectos críticos: el SIM800L presenta picos de consumo que pueden superar los 2 A durante la transmisión, se  dispone de una fuente con suficiente margen y filtros adecuados para evitar reinicios inesperados.
%
%\item Limitaciones: ancho de banda reducido (máximo teórico de 85,6 kbps en GPRS), lo que refuerza la decisión de emplear MQTT por su bajo overhead.
%
%\end{itemize}
%
%
%\subsection{Contador de tránsito y comunicación RS-232}
%
%El sistema de conteo existente genera tramas con información de los eventos de paso de vehículos acumulados en un intervalo de tiempo (por defecto 1 hora). La comunicación con el ESP32-C3 se establece mediante interfaz RS-232, estándar ampliamente utilizado para transmisión serial de datos.
%
%\begin{itemize}
%\item Formato de trama:  el equipo incluye identificador de Dipositivo, timestamp local, valores de clasificación tránsito por carril.
%
%\item Adaptación eléctrica: se utiliza un conversor de niveles (MAX232) para adaptar las señales RS-232 al rango TTL del microcontrolador.
%
%\item Ventaja: la reutilización de la interfaz serial del contador evita modificar el sistema de detección existente, reduciendo costos de integración.
%
%
%\end{itemize}
%
%
%En la figura \ref{fig:foto_dtec} se observa el contador de tránsito. Este dispositivo constituye la base del sistema de detección de eventos y se mantiene sin modificaciones en su lógica interna. 
%
%
%\begin{figure}[htbp]
%  \centering
%  \includegraphics[width=0.5\linewidth]{./Figures/fotoDTEC.jpeg}
%  \caption{Contador de tránsito DTEC \protect\footnotemark.}
%  \label{fig:foto_dtec}
%\end{figure}
%
%\footnotetext{Imagen tomada de \url{http://transito.vialidad.gob.ar/}}

\item Alimentación y montaje: para garantizar el funcionamiento continuo del nodo en entornos de campo,
se deben considerar dos aspectos fundamentales:
\begin{itemize}
\item Fuente de alimentación: se implementó un sistema de energía basado en una batería interna recargable, específicamente dimensionada para cubrir los picos de consumo del módulo SIM800L durante las transmisiones de datos. La autonomía del sistema está garantizada mediante un panel solar que mantiene la carga de la batería de forma continua.

Para estabilizar la entrega de energía se incorporó un módulo regulador de tensión que garantiza el nivel adecuado de voltaje para el módem. El circuito se complementó con capacitores dimensionados para absorber los picos de tensión del SIM800L, evitando caídas de voltaje que puedan reiniciar el sistema.

La figura \ref{fig:diag_conexiones} se presenta el diagrama de conexión entre los módulos del sistema, detallando las líneas de comunicación serie RS232, la interfaz UART entre el microcontrolador y el módem GSM, así como la distribución de la alimentación eléctrica y los circuitos de estabilización de potencia.

\begin{figure}[H]
  \raggedleft
  \includegraphics[width=0.8\linewidth]{./Figures/diagConexion.png}
  \caption{Diagrama de conexión entre los módulos del sistema.}
  \label{fig:diag_conexiones}
\end{figure}



\item Carcasa y gabinete: tanto el contador como el módulo de comunicación remota (ESP32-C3 y SIM800L) cuentan con su propia carcasa de protección y, adicionalmente, ambos se alojan en un gabinete metálico estandarizado ya existente en la infraestructura vial, diseñado para resistir humedad, polvo y vibraciones. Esta solución aprovecha la protección ambiental, disipación térmica y blindaje electromagnético del gabinete original, que garantiza la confiabilidad del sistema en condiciones de intemperie.

En la figura \ref{fig:foto_gabinete} se observa el contador de tránsito instalado en campo, junto con su gabinete de protección y la batería interna que asegura autonomía energética. El conjunto se encuentra montado al costado de la ruta, en condiciones reales de operación, lo que permite apreciar el encapsulado diseñado para resistir las exigencias ambientales del entorno vial.


\begin{figure}[H]
  \centering
  \includegraphics[width=1\linewidth]{./Figures/fotoGabinete.jpeg}
  \caption{Fotografia contador de tránsito DTEC instalado en campo \protect\footnotemark.}
  \label{fig:foto_gabinete}
\end{figure}

\end{itemize}
\footnotetext{Imagen tomada de \url{http://transito.vialidad.gob.ar/}}

\end{itemize}

\section{Desarrollo del backend}

El backend es el núcleo lógico del sistema, encargado de integrar dispositivos, base de datos e interfaz. Su diseño prioriza la modularidad, escalabilidad y seguridad, con tecnologías comunes en entornos IoT.


\subsection{Arquitectura y tecnologías}

El servicio se implementó en Node.js con Express, organizando la aplicación en controladores, rutas y middlewares, y usando Sequelize \cite{sequelize}, un ORM \cite{fowler2002patterns} que facilita los modelos y asegura independencia de la persistencia.

La comunicación con los dispositivos se realiza mediante tópicos MQTT en Eclipse Mosquitto.

Las mediciones se publican en tópico \texttt{dispositivo/{id}/medicion}, mientras que el backend se encarga de validarlas y almacenarlas en MySQL. Por su parte, los comandos y respuestas se gestionan mediante los tópicos \texttt{dispositivo/{id}/comando} y \texttt{dispositivo/{id}/respuesta}, respectivamente. El despliegue del sistema se realiza con  Docker Compose \cite{docker_compose}, que permite ejecutar backend, base de datos y broker \cite{mqttSpec} en contenedores independientes. Además, el registro y la trazabilidad se gestionan con Winston \cite{winston} y Morgan \cite{morgan}, lo que garantiza un monitoreo completo del sistema

En la figura \ref{fig:diagrama_backend} se observa el diagrama de flujo de información del backend.


\begin{figure}[H]
 
  \centering
  \includegraphics[width=0.9\linewidth]{./Figures/diagFlujoConexionBackend.png}
  \captionof{figure}{Diagrama de flujo de información del Backend.}
  \label{fig:diagrama_backend}
  \end{figure}


\subsection{Funcionalidades principales}
El sistema cuenta con varias funcionalidades esenciales que permiten gestionar de manera eficiente los dispositivos, los eventos generados por ellos, los comandos enviados y la seguridad de acceso. Estas funcionalidades se detallan a continuación:

\begin{itemize}
    \item Gestión de dispositivos: alta, baja, modificación y consulta.
    \item Gestión de eventos: almacenamiento de detecciones y consultas filtradas por dispositivo o rango temporal.
    \item Gestión de comandos: emisión de órdenes a un dispositivo, persistencia de la orden con identificador único (cmd\_id) y actualización según respuesta.
    \item Estado de dispositivos: consulta de parámetros como nivel de batería, temperatura o conectividad.
    \item Autenticación y autorización: control de acceso mediante tokens JWT.
\end{itemize}


\subsection{Organización en controladores}

La lógica de negocio del backend se organiza en controladores, cada uno asociado a un recurso del sistema, lo que favorece la separación de responsabilidades, el mantenimiento y la escalabilidad. Los principales controladores son:

\begin{itemize}
  \item DispositivoController: gestiona las operaciones CRUD sobre los dispositivos de campo, además de registrar los eventos recibidos vía MQTT y asociarlos a un dispositivo específico.
  \item MedicionController: encapsula la lógica de ingesta de eventos de tránsito, validación de payloads y persistencia en la base de datos.
  \item ComandoController: administra la emisión y seguimiento de comandos remotos, generando un \texttt{cmd\_id} único.
  
  \item RespuestaController: centraliza la recepción de estados y telemetría (batería, conectividad), en respuesta al comando que se envía, esto garantiza que la base de datos refleje la situación en tiempo real.
  \item UserController: implementa el ciclo de vida de usuarios y la autenticación mediante JWT\cite{jwtRFC7519}, así como la validación de permisos en cada endpoint.
\end{itemize}


En la figura \ref{fig:diagrama_controladores} se observa el diagrama con la disposición de los controladores y flujo de dependencias.

\begin{figure}[H]
 
  \centering
  \includegraphics[scale=0.146]{./Figures/diagDispoControlladores.png}
  \captionof{figure}{Diagrama con la disposición de los controladores y flujo de dependencias.}
  \label{fig:diagrama_controladores}
  \end{figure}

\subsection{Mapa de endpoints}

El backend expone una serie de endpoints REST que conforman la interfaz principal de comunicación con los servicios de aplicación y los dispositivos de campo.  
En la tabla \ref{tab:endpoints} se presentan los endpoints generales.

\begin{table}[H]
	\centering
	\caption[Endpoints REST principales]{Endpoints REST principales expuestos por el backend, junto con el controlador que implementa su lógica.}
	\begin{tabular}{l l p{3.4cm}}    
		\toprule
		\textbf{Endpoint} & \textbf{Controlador} & \textbf{Descripción} \\
		\midrule
		GET /dispositivo & DispositivoController & Lista todos los dispositivos registrados \\
		GET /dispositivo/\{id\} & DispositivoController & Devuelve información de un dispositivo específico \\
		POST /dispositivo & DispositivoController & Alta de un nuevo dispositivo \\
		PATCH /dispositivo/\{id\} & DispositivoController & Actualización de atributos de un dispositivo \\
		DELETE /dispositivo/\{id\} & DispositivoController & Eliminación de un dispositivo \\
		\addlinespace
		POST /medicion & MedicionController & Crea  mediciones de un dispositivo \\
		GET /medicion/dispositivo/\{id\} & MedicionController & Consultar mediciones por dispositivo \\
		GET /medicion/range & MedicionController & Consultar mediciones por rango temporal \\
		\addlinespace
		POST /comando & ComandoController & Crear un comando remoto y publicarlo en MQTT \\
		GET /comando/\{id\} & ComandoController & Consultar un comando  \\
		\addlinespace
		GET /respuesta/\{id\} & RespuestaController & Consultar respuesta de un comando \\
		\addlinespace
		POST /usuario/login & UserController & Autenticación de usuario, devuelve token JWT \\
		POST /usuario & UserController & Alta de usuario \\
		GET /usuario & UserController & Listar usuarios registrados \\
		DELETE /usuario/\{id\} & UserController & Eliminar usuario \\
		\bottomrule
		\hline
	\end{tabular}
	\label{tab:endpoints}
\end{table}


\subsection{Seguridad y extensibilidad}

Además de la autenticación mediante JWT, todos los endpoints aplican validaciones y sanitización de parámetros de entrada y salida. El sistema de logging, implementado con Winston y Morgan, garantiza trazabilidad de las operaciones tanto en la capa HTTP como en la mensajería MQTT. La arquitectura modular basada en controladores permite extender el backend con nuevos recursos o funcionalidades sin afectar la lógica ya implementada.


\section{Desarrollo del frontend}

El frontend, desarrollado como \textit{Single Page Application} en Ionic con Angular y TypeScript, ofrece una interfaz moderna y responsiva que permite autenticación, supervisión en tiempo real, consulta de históricos y envío de comandos a los nodos.

\subsection{Arquitectura y tecnologías}

La aplicación se compone de módulos reutilizables de Ionic, optimizados para escritorio y móviles.
La comunicación con el backend se realiza mediante API REST y, para actualizaciones en tiempo real, a través de WebSocket.


\subsection{Funcionalidades principales}

El frontend integra las siguientes funciones clave:
\begin{itemize}
    \item Login de usuario: ingreso con credenciales, validación contra la API y obtención de un token JWT.
    \item Listado de dispositivos: muestra todos los contadores registrados, junto con información de ubicación y estado básico.
    \item Detalle de dispositivo: despliega datos específicos de un contador y últimas tramas recibidas.
    \item Panel de mediciones: permite visualizar los eventos de tránsito procesados, con actualización dinámica cuando el dispositivo transmite nuevas tramas.
    \item Historial de eventos: consulta de registros almacenados en la base de datos, filtrados por dispositivo y rango temporal.
  \item Envío de comandos: permite emitir órdenes remotas al nodo (reset, cambio u obtención de parámetros, etc.). El sistema verifica el acuse de recibo y muestra el resultado (ok, failed, timeout o value).   
\end{itemize}

En la figura \ref{fig:diagrama_front_pantallas} se observa la estructura de los componentes por pantalla.
\begin{figure}[H]
 
  \centering
  \includegraphics[width=0.8\linewidth]{./Figures/diagEstructuraComponentesPantallasFrontend.png}
  \captionof{figure}{Diagrama de estructura de los componentes por pantalla.}
  \label{fig:diagrama_front_pantallas}
\end{figure}



\subsection{Integración con el backend}

El frontend utiliza los endpoints REST del backend (ver Sección~\ref{tab:endpoints}), enviando en cada petición el token JWT obtenido en el login para asegurar el acceso autorizado. Las respuestas JSON se interpretan en tiempo real, manteniendo la interfaz sincronizada con el estado de los dispositivos.


En la figura \ref{fig:diagramaflujorestapi} se observa el diagrama de flujo de comunicación con el backend.

\begin{figure}[H]
 
  \centering
  \includegraphics[width=1\linewidth]{./Figures/diagFlujoRestApi.png}
  \captionof{figure}{Diagrama de flujo de comunicación con el backend.}
  \label{fig:diagramaflujorestapi}
\end{figure}

\section{Despliegue del sistema}

El despliegue del sistema comprende la puesta en marcha coordinada de los distintos servicios que componen la arquitectura: el broker MQTT, la API REST, la base de datos relacional y la interfaz web.  
El objetivo es trasladar el prototipo desde un entorno de desarrollo hacia un entorno productivo, que asegura escalabilidad, confiabilidad y capacidad de monitoreo post-implantación.  

\subsection{Entorno productivo e integración continua}

El entorno productivo se implementa bajo un esquema de \textit{cloud on-premise}, es decir, una nube privada alojada en los servidores locales de Vialidad Nacional. 
Este enfoque combina las ventajas de la virtualización y la gestión centralizada propias del entorno \textit{cloud}, con el control, la seguridad y la independencia de un despliegue local.

Para la orquestación se emplea Docker Compose, que permite instanciar todos los servicios (broker MQTT, API REST, base de datos y aplicación web) en contenedores aislados pero comunicados entre sí. 
De esta forma, el sistema puede escalar, actualizarse y mantenerse de manera unificada, preservando la trazabilidad y la integridad de los datos.

La integración continua (CI) automatiza la construcción, prueba y despliegue del sistema. 
Cada actualización del repositorio genera nuevas imágenes Docker, ejecuta validaciones automáticas y despliega los servicios en el entorno \textit{on-premise}, que garantiza coherencia entre versiones y reduciendo errores manuales.

\subsection{Monitoreo post-implantación}

Una vez desplegado el sistema, resulta fundamental contar con mecanismos de monitoreo que permitan evaluar su correcto funcionamiento en campo:  

\begin{itemize}
    \item Logs centralizados: tanto el backend como el broker MQTT registran eventos en archivos y consola. Se  integra con Grafana para correlacionar métricas.
    
    \item Alertas y métricas: mediante Grafana es posible recolectar indicadores de CPU, memoria y estado de contenedores. También se pueden graficar métricas de tráfico MQTT (mensajes publicados, latencias, pérdidas).
    
    \item Supervisión de dispositivos: la API REST expone endpoints que informan conectividad y parámetros básicos (nivel de batería, último evento recibido). Estos datos se representan en la interfaz web como panel de salud del sistema.
    
    \item Respaldo y recuperación: la base de datos implementa backups automáticos y permite restauraciones parciales. Esto garantiza que el historial de eventos no se pierda ante fallas de hardware o corrupción de datos.
\end{itemize}


\section{Integración con la infraestructura existente}
Una de las principales ventajas de este diseño es que no requiere modificaciones internas en el contador de tránsito. El nodo recibe los pulsos de detección mediante la interfaz RS-232, que preserva la integridad del equipo original. 

El ESP32-C3 no se limita a reenviar datos, sino que añade valor al sistema al realizar un preprocesado local: filtra tramas, agrupa eventos en función de ventanas de tiempo y asegura la transmisión con políticas de reintento. Asimismo, la conexión con el servidor central mediante MQTT garantiza interoperabilidad con aplicaciones externas y facilita la escalabilidad del sistema.

En este contexto, los nodos de campo cumplen un doble rol: por un lado, son captadores de datos provenientes de los sensores de tránsito y por otro, actúan como puntos de control remoto, capaces de ejecutar comandos enviados desde la plataforma central. Esta dualidad refuerza la flexibilidad del sistema y lo hace adaptable a distintas políticas de gestión vial.




