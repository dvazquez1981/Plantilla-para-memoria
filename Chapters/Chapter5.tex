\chapter{Conclusiones}
\label{Chapter5}

En este capítulo se presentan las conclusiones generales del trabajo, junto con una reflexión sobre los resultados alcanzados y las posibles líneas de desarrollo futuro.  

%----------------------------------------------------------------------------------------
\section{Resultados y metas}
\label{sec:resultados_metas}

El sistema desarrollado permitió cumplir los objetivos propuestos, logra una solución integral para la detección y transmisión de eventos de tránsito en entornos con conectividad limitada.  
La arquitectura implementada demostró ser eficiente, modular y adaptable a distintos escenarios de despliegue, que mantiene la integridad de los datos y la estabilidad del flujo de comunicación entre nodos de campo, servidor y cliente web.

A continuación, se sintetizan los principales resultados y aportes del trabajo:

\begin{itemize}
  \item Se logró diseñar e implementar un sistema distribuido basado en protocolos abiertos (MQTT, REST y JSON), capaz de operar de forma confiable ante conectividad intermitente.  

  \item El firmware embebido en el ESP32-C3 validó su capacidad para procesar tramas RS-232, almacenar eventos temporalmente y asegurar su retransmisión una vez restablecida la conexión.  

  \item Se desarrolló una biblioteca específica para el manejo del módulo SIM800L, que permite ejecutar comandos AT y establecer la comunicación con el broker MQTT y que garantiza la publicación y suscripción de mensajes bajo condiciones variables de red.  
  \item El backend, desarrollado en Node.js con Express y MySQL, garantizó la persistencia de los datos, la trazabilidad de los eventos y la autenticación segura mediante tokens JWT.  
  
  \item El broker MQTT (Eclipse Mosquitto) permitió la comunicación asincrónica y la publicación confiable de mensajes entre dispositivos y servidor, con soporte para retención de mensajes.  
  
  \item La interfaz web, desarrollada en Ionic y Angular, proporcionó una visualización clara y funcional de los eventos de tránsito, junto con la posibilidad de emitir comandos y supervisar el estado de los nodos en tiempo real.  
  
  \item Las pruebas realizadas en laboratorio  demostraron la estabilidad del sistema y su capacidad de recuperación ante fallas o desconexiones.  
  
  \item La arquitectura modular permitió aislar componentes, facilitar el mantenimiento y posibilitar futuras extensiones sin comprometer la estabilidad del sistema base.
\end{itemize}

El cronograma original se mantuvo en líneas generales, con ajustes menores asociados a la disponibilidad de hardware y a la calibración de los tiempos de respuesta del módulo GPRS.  
Los riesgos identificados durante la planificación, vinculados principalmente a la inestabilidad del enlace móvil y a la gestión de colas locales, fueron mitigados mediante estrategias de reconexión automática y verificación de integridad de datos, que demostraron ser efectivas en las pruebas finales.


%----------------------------------------------------------------------------------------
\section{Trabajo futuro}
\label{sec:trabajo_futuro}

El trabajo realizado es un punto de partida para la evolución del sistema hacia una plataforma más completa, eficiente y adaptable a entornos operativos reales.  
Entre las líneas de continuidad propuestas se destacan las siguientes:

\begin{itemize}
  \item Incorporar un módulo de comunicación que integre nativamente el protocolo MQTT, en reemplazo del SIM800L. Esta modificación permitiría reducir la complejidad del firmware, aumentar la confiabilidad del enlace y mejorar los tiempos de transmisión.  

  \item Implementar mecanismos de actualización remota OTA para el firmware de los nodos de campo, con el fin de simplificar el mantenimiento y asegurar la uniformidad de versiones en despliegues múltiples.  
  
 \item Incorporar un módulo de análisis histórico y visualización avanzada que permita detectar patrones de tránsito, generar reportes automáticos y apoyar la toma de decisiones operativas.  
 
  \item Integrar servicios de geolocalización y mapas interactivos para representar la ubicación de los dispositivos y eventos en tiempo real.  
 
  \item Evaluar el desempeño del sistema en entornos de campo prolongados, con el propósito de obtener métricas de confiabilidad, consumo energético y latencia bajo condiciones reales de operación.  

  \item Desarrollar herramientas de monitoreo remoto y alertas automáticas, que notifiquen anomalías en los dispositivos o interrupciones de comunicación sin intervención manual.  

  \item Incorporar técnicas de inteligencia artificial para la detección de tránsito no registrado, lo que permitiría incrementar la precisión del sistema en escenarios con oclusión parcial o interferencias.  
  \item Migrar la comunicación móvil a redes 4G o 5G, con el fin de mejorar la disponibilidad, reducir la latencia y soportar mayor volumen de datos en despliegues masivos.  
\end{itemize}


