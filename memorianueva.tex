%----------------------------------------------------------------------------------------
% Preambulo y Configuración
%----------------------------------------------------------------------------------------
\PassOptionsToPackage{backend=biber,style=ieee}{biblatex}
\documentclass[
    11pt,
    spanish,
    singlespacing,
    parskip,
    headsepline,
    bookmarks=true,
    unicode=true,
    pdftoolbar=true,
    pdfmenubar=true,
    pdffitwindow=false,
    colorlinks=true,
    linkcolor=blue,
    citecolor=blue,
    urlcolor=blue,
    biblatexoptions={backend=biber,style=ieee}
]{MastersDoctoralThesis}

\usepackage[utf8]{inputenc} % Codificación de entrada UTF-8
\usepackage[T1]{fontenc}    % Codificación de salida para caracteres especiales
\usepackage{graphicx}       % Manejo de gráficos
\usepackage{eso-pic}        % Permite agregar fondos
\usepackage{hyperref}       % Manejo de hipervínculos y marcadores
\usepackage{pdflscape}
\usepackage{placeins}
\usepackage{float}

%\usepackage[backend=biber,style=ieee]{biblatex}
\addbibresource{bibliografia.bib}



% Redefinición de caracteres problemáticos en marcadores
\hypersetup{
    pdftitle={Título del Documento},
    pdfauthor={Autor del Documento},
    pdfkeywords={Sistemas Embebidos, Internet de las Cosas, Inteligencia Artificial},
    pdfstartview={FitH},
    unicode=true,
    colorlinks=true,
    linkcolor=blue,
    citecolor=blue,
    urlcolor=blue
}

\pdfstringdefDisableCommands{%
  \def\texttt#1{#1}%
  \def\textbf#1{#1}%
  \def\textit#1{#1}%
  \def\textunderscore{\_}%
  \def\"#1{#1}%
  \def\~#1{#1}%
  \def\'#1{#1}%
}



% Definir comandos requeridos por la clase
\newcommand{\degreename}{Especialidad en Internet de las cosas} % Cambia según tu título
\newcommand{\univname}{Universidad de Buenos} % Cambia según tu universidad
\newcommand{\keywordnames}{Palabras clave:}
%----------------------------------------------------------------------------------------
% Documento Principal
%----------------------------------------------------------------------------------------

\begin{document}

% Configuración de la portada
\posgrado{Carrera / Maestría}
\keywords{Sistemas Embebidos, Internet de las Cosas, Inteligencia Artificial}

% Incluir la portada desde un archivo separado
\include{portada}

% Configuración del contenido preliminar
\frontmatter % Usar numeración romana para las páginas preliminares
\pagestyle{plain} % Estilo de encabezado simple

%----------------------------------------------------------------------------------------
% Resumen
%----------------------------------------------------------------------------------------

\begin{abstract}
\addchaptertocentry{\abstractname} % Agregar resumen al índice
Esta memoria describe el desarrollo e implementación de un sistema para el registro de eventos de tránsito y la transmisión segura de datos. El diseño asegura la disponibilidad de la información incluso ante interrupciones en la conexión a Internet. El sistema fortalece el monitoreo de las rutas nacionales mediante comunicación bidireccional. Permite realizar diagnósticos remotos, actualizar parámetros, incrementar la precisión y consistencia de los datos y optimizar el trabajo del personal técnico y del equipamiento de monitoreo. Para su realización se aplicaron conocimientos de Internet de las cosas, transmisión segura de datos y control de sistemas remotos.
\end{abstract}

%----------------------------------------------------------------------------------------
% Agradecimientos
%----------------------------------------------------------------------------------------

\begin{acknowledgements}
\vspace{1.5cm}
Quiero expresar mi más profundo agradecimiento a mi familia, quienes han sido un pilar fundamental en cada etapa de mi vida.  
\end{acknowledgements}

%----------------------------------------------------------------------------------------
% Índice
%----------------------------------------------------------------------------------------

\tableofcontents
\listoffigures
\listoftables

%----------------------------------------------------------------------------------------
% Dedicatoria
%----------------------------------------------------------------------------------------



%----------------------------------------------------------------------------------------
% Capítulos
%----------------------------------------------------------------------------------------

\mainmatter % Iniciar numeración numérica para el contenido principal
\pagestyle{thesis} % Estilo de encabezado de tesis

% Incluir capítulos desde archivos separados
% Chapter 1

\chapter{Introducción general} % Main chapter title

\label{Chapter1} % For referencing the chapter elsewhere, use \ref{Chapter1} 
\label{IntroGeneral}

%----------------------------------------------------------------------------------------

% Define some commands to keep the formatting separated from the content 
\newcommand{\keyword}[1]{\textbf{#1}}
\newcommand{\tabhead}[1]{\textbf{#1}}
\newcommand{\code}[1]{\texttt{#1}}
\newcommand{\file}[1]{\texttt{\bfseries#1}}
\newcommand{\option}[1]{\texttt{\itshape#1}}
\newcommand{\grados}{$^{\circ}$}

En este capítulo se presenta el marco de referencia y la justificación del trabajo. Se expone el contexto que motivó su desarrollo, los problemas detectados en la infraestructura actual, una revisión del estado del arte, la propuesta de valor y el alcance del prototipo planteado.


%----------------------------------------------------------------------------------------

%\section{Introducción}

%----------------------------------------------------------------------------------------

\section{Motivación}

Este trabajo propone la modernización de los contadores de tránsito instalados en rutas nacionales mediante la renovación de su arquitectura de comunicaciones. 
\subsection{Contexto actual}
Actualmente, los equipos registran el paso de vehículos y transmiten eventos al servidor central a través de enlaces GPRS tercerizados. No obstante, no admiten la recepción de comandos ni la obtención de diagnósticos remotos. Esta limitación reduce la capacidad operativa, incrementa los costos de mantenimiento y demora la resolución de fallas, debido a que todo ajuste o reparación requiere una intervención presencial \cite{asiain2021lora}, \cite{micko2023iot}.
\subsection{Limitaciones del desarrollo previo}
La propuesta se origina a partir del desarrollo de un contador de tránsito destinado a registrar y transmitir información en campo. Sin embargo, la conexión remota de este dispositivo presenta limitaciones en cuanto a su capacidad de transmisión de datos y carece de mecanismos de control remoto, lo que dificulta tanto la supervisión del funcionamiento como la actualización de sus parámetros. Frente a estas restricciones, surge la necesidad de modernizar el sistema mediante la incorporación de una comunicación bidireccional confiable \cite{peruzzi2022lorawan}, \cite{micko2023iot}. 

El análisis del sistema vigente permitió identificar las siguientes limitaciones que motivan el rediseño:

\begin{itemize}
\item  Comunicación unidireccional. Los contadores envían datos al servidor, pero no existe un canal para enviar configuraciones, consultas y comandos desde el servidor hacia los equipos. Esta limitación impide realizar diagnósticos remotos y ejecutar acciones correctivas sin presencia física.

\item  Dependencia de proveedores GPRS tercerizados. La dependencia de servicios contratados genera costos recurrentes y limita el control sobre la calidad y disponibilidad de la conectividad.

\item  Imposibilidad de actualización remota. Cualquier modificación de parámetros o ajustes de operación requiere intervención en el sitio. Esto incrementa tiempos de mantenimiento, costos logísticos y complica la aplicación rápida de mejoras.

\item  Falta de telemetría y diagnóstico preventivo. No se dispone de métricas sistemáticas sobre el estado operativo de los equipos (batería, temperatura, errores de hardware o comunicación).

\item Riesgo de pérdida de datos ante conectividad intermitente. La ausencia de mecanismos de encolamiento persistente y de políticas claras de reenvío eleva la probabilidad de pérdida o duplicación de eventos cuando la red es inestable.

\end{itemize}

Estas deficiencias afectan la calidad del servicio de monitoreo, reducen la eficiencia operativa y constituyen los requisitos funcionales que orientan el diseño del prototipo.




\subsection{Impacto esperado}
La modernización permite reducir los desplazamientos de personal técnico y acortar los tiempos de resolución de incidentes con la consiguiente disminución de costos logísticos. La incorporación de telemetría facilita la planificación de intervenciones preventivas en lugar de responder exclusivamente a fallas, lo que optimiza la disponibilidad del servicio y la calidad de los datos recolectados. Además, la adopción de protocolos estandarizados y componentes de código abierto promueve la escalabilidad y la replicabilidad de la solución en distintos tramos de la red vial \cite{miovision},\cite{sensys}, \cite{metrocount}.
\subsection{Diseño conceptual} 
La propuesta separa de forma explícita el transporte de mensajes (broker MQTT) de los servicios de aplicación (API REST, almacenamiento y frontend). Esta separación facilita la interoperabilidad con plataformas institucionales existentes y habilita opciones de despliegue flexibles: uso de brokers externos, instalación de brokers locales o modelos híbridos según políticas institucionales y condiciones de conectividad.

\section{Objetivos}

\subsection{Objetivo general}

Modernizar los contadores de tránsito en rutas nacionales mediante la renovación de su arquitectura de comunicaciones para habilitar comunicación bidireccional confiable, control remoto y telemetría de estado.

\subsection{Objetivos específicos}

A continuación, se detallan los objetivos específicos que orientan el desarrollo del trabajo:
\begin{itemize}

\item  Garantizar la entrega de eventos aun con conectividad intermitente.

\item Implementar mecanismos de encolado y reintento que eviten duplicaciones y pérdidas de información.

\item  Habilitar la ejecución remota de comandos y la actualización de parámetros desde un servidor central, con confirmación explícita del estado del equipo.

\item  Disminuir la dependencia de enlaces tercerizados mediante una arquitectura configurable.

\item  Permitir la modificación remota de parámetros operativos y la carga de ajustes sin desplazamientos.

\item Incorporar telemetría de estado (batería, temperatura y códigos de error) para mantenimiento preventivo.

\item Validar la viabilidad técnica y la robustez operativa del sistema.
\end{itemize}


\section{Estado del arte y propuesta de valor}

En el mercado existen soluciones comerciales \cite{exemys}, \cite{digiRemoteManager}, que ofrecen gestión remota y comunicación bidireccional para dispositivos de campo. Dichas soluciones suelen incluir plataformas propietarias con soporte técnico, servicios administrados y herramientas de análisis avanzadas. Estas alternativas, sin embargo, presentan costos elevados de adquisición y mantenimiento, así como dependencias tecnológicas que restringen la posibilidad de adaptación a condiciones locales específicas.

En el ámbito académico y técnico también se han documentado diversas experiencias orientadas a la gestión remota de infraestructuras distribuidas mediante protocolos de mensajería ligera como MQTT o tecnologías de comunicación celular \cite{monitoringVehicles2023}, \cite{iotSmartTraffic2021}. Estos trabajos resaltan la eficiencia de los modelos de publicación y suscripción, especialmente en entornos con restricciones de conectividad, pero en muchos casos se centran en aplicaciones industriales que difieren de las condiciones propias de las rutas nacionales \cite{iopMQTTSystem2023}.

Además, se han desarrollado plataformas de monitoreo basadas en API REST y bases de datos relacionales, que permiten la integración con interfaces web para la visualización y control \cite{openRemoteSolution}. Estas experiencias muestran la tendencia hacia arquitecturas abiertas y modulares, aunque su adopción práctica suele estar ligada a contextos con mayor disponibilidad de infraestructura y recursos.

El estado del arte muestra soluciones maduras para la gestión remota y la comunicación bidireccional, aunque con limitaciones asociadas a costos elevados, rigidez tecnológica o requerimientos de infraestructura. Estas restricciones evidencian la necesidad de alternativas específicas y adaptadas a entornos viales argentinos, donde la conectividad suele ser intermitente.


\section{Alcance}
El trabajo desarrolla un prototipo funcional destinado a validar las hipótesis técnicas y operativas. El alcance comprende los siguientes objetivos y límites:

\begin{itemize}

\item Rediseño de comunicaciones: implementación de un modelo bidireccional y seguro entre los dispositivos de conteo y el servidor central, basado en MQTT sobre GPRS y con autenticación por credenciales.

\item Gestión de mensajes en el dispositivo: encolamiento en memoria RAM con política FIFO, control de reintentos ante fallos de conexión y lógica de descarte cuando la cola alcance su límite definido.

\item Backend y persistencia: desarrollo de una API REST que se suscriba al broker MQTT, procese los eventos y almacene los registros en una base de datos relacional para consulta histórica.

\item Interfaz web básica: panel de visualización en tiempo real de eventos de tránsito y módulo para envío de comandos y verificación de estado de los dispositivos.

\item Funciones de telemetría: reporte de nivel de batería, temperatura y códigos de error para facilitar el diagnóstico remoto.

\end{itemize}

El alcance técnico incluye la adaptación de un contador existente para comunicarse bidireccionalmente por GPRS, que emplee MQTT como protocolo de mensajería ligera y confiable. Asimismo, comprende el desarrollo de un backend que reciba, procese y persista los eventos en una base de datos relacional. Además, una interfaz web básica que permita la visualización en tiempo real y el envío de comandos hacia los dispositivos. Se priorizan las funciones que validen la cadena completa: adquisición de eventos desde el sistema de detección, encolamiento local con política FIFO, reenvío seguro cuando haya conectividad, recepción y ejecución de comandos y verificación del estado del dispositivo tras cada acción. 


La figura \ref{fig:diag_bloques} muestra el diagrama en bloques del sistema. 
El dispositivo de conteo envía datos por GPRS a un broker MQTT, el servidor central 
los procesa y almacena en una base de datos, mientras que una API REST permite su 
consulta y la ejecución de comandos desde una interfaz web.

\begin{figure}[H]
  \centering
  \includegraphics[width=\linewidth]{./Figures/diagBloques.png}
  \caption{Diagrama en bloques del sistema.}
  \label{fig:diag_bloques}
\end{figure}













\chapter{Introducción específica} % Main chapter title

\label{Chapter2}

%----------------------------------------------------------------------------------------
%	SECTION 1
%----------------------------------------------------------------------------------------
En este capítulo se detallan los aspectos técnicos específicos que constituyen la base del trabajo. En primer lugar, se describen los protocolos de comunicación que se emplean y la función de cada uno de ellos en la transmisión de datos. Luego, se presentan los componentes de hardware que utiliza el prototipo. A continuación, se analizan las tecnologías de software que integran la solución, la herramienta de control de versiones adoptada para el desarrollo colaborativo y la gestión del código fuente.

\section{Protocolos y comunicación}
El diseño de un sistema de conteo de tránsito con comunicación bidireccional demanda la incorporación de protocolos que aseguren confiabilidad y eficiencia en el intercambio de datos. En este trabajo se integran tres tecnologías principales: 

\begin{itemize}
	\item RS-232: es un estándar de comunicación serial utilizado tradicionalmente en sistemas embebidos. Permite la transmisión de datos punto a punto entre el contador de tránsito y el microcontrolador ESP32-C3 \cite{esp32c3}. Su simplicidad lo hace adecuado para distancias cortas y ambientes donde la interferencia es controlada. Aunque se trata de un protocolo clásico, su adopción garantiza compatibilidad con dispositivos que aún dependen de interfaces seriales \cite{analogRS232},  \cite{tiRS232}.
	
	\item MQTT: este protocolo de mensajería ligera se emplea tanto para la transmisión de eventos de tránsito desde los dispositivos hacia el servidor central como para la recepción de comandos en sentido inverso. MQTT opera sobre TCP/IP \cite{comerTCPIP} y emplea un modelo de publicación/suscripción a través de un broker, lo que facilita la escalabilidad y la integración de múltiples dispositivos. Además, permite implementar mecanismos de calidad de servicio (QoS) que reducen la probabilidad de pérdida de datos en condiciones de conectividad inestable, lo que resulta crucial para el escenario de rutas nacionales \cite{mqttSpec}.
	
	\item API REST: constituye la interfaz de comunicación entre el backend y los clientes web. Su inclusión en el trabajo posibilita la consulta y el envío de información de manera estructurada, mediante operaciones estándar (GET, POST, PUT, DELETE). La API REST asegura la interoperabilidad con diferentes plataformas y brinda flexibilidad para desarrollar aplicaciones adicionales que utilicen los datos recolectados \cite{ibmRest}, \cite{microsoftApiDesign} .
		
\end{itemize}

La combinación de estos protocolos permite cubrir distintos niveles de la arquitectura: comunicación local (RS-232), comunicación de dispositivos con el servidor (MQTT) y comunicación entre el servidor y las aplicaciones de usuario (API REST). De esta forma, se garantiza un flujo de datos seguro, confiable y bidireccional.



%----------------------------------------------------------------------------------------

\section{Componentes de hardware utilizados}

El prototipo se implementa con un conjunto de componentes de hardware seleccionados por su disponibilidad, costo y adecuación al entorno de operación:

\begin{itemize}

\item  Contador de tránsito: dispositivo de campo encargado de detectar y acumular el paso de vehículos y generar tramas de datos que contienen información de conteo, clasificación y marcas temporales. Estos equipos permiten registrar de manera continua el flujo vehicular en rutas nacionales y constituyen la base del sistema de monitoreo. Si bien en el mercado existen distintos contadores comerciales que ofrecen funcionalidades similares, muchos de ellos presentan altos costos de adquisición, dependencia de plataformas propietarias o limitaciones de integración. En este trabajo se optó por utilizar un contador desarrollado por la Dirección Nacional de Vialidad, que ha sido probado en distintos rutas nacionales del país y cuenta con una interfaz de salida RS-232 \cite{tiRS232} con un adaptador MAX232 \cite{max232}. Esta elección responde a criterios de soberanía tecnológica, optimización de recursos ya disponibles y reducción de costos. Además, el contador no solo transmite información de eventos, sino que también admite la recepción de comandos externos a través del nodo, lo que habilita funcionalidades de diagnóstico, reconfiguración remota y confirmación de estado. De este forma, se garantiza la compatibilidad con la infraestructura existente y se potencia su integración dentro de una arquitectura moderna basada en protocolos abiertos.

En la figura \ref{fig:foto_dtec2} se observa el contador de tránsito. Este dispositivo constituye la base del sistema de detección de eventos. 

\begin{figure}[H]
  \centering
  \includegraphics[width=0.4\linewidth]{./Figures/fotoDTEC2.png}
  \caption{Contador de tránsito DTEC \protect\footnotemark.}
  \label{fig:foto_dtec2}
\end{figure}
\footnotetext{Imagen tomada de \url{http://transito.vialidad.gob.ar/}}

 \item Módulo de adaptación RS-232/TTL (MAX232): este circuito se encarga de convertir los niveles de tensión de forma bidireccional, protege los dispositivos y garantiza una transmisión de datos confiable y libre de errores \cite{max232}.
 
  En la figura \ref{fig:foto_max232} se observa el módulo de adaptación RS-232/TTL (MAX232).


\begin{figure}[H]
  \centering
  \includegraphics[width=0.4\linewidth]{./Figures/fotoMax232.png}
  \caption{Módulo RS-232/TTL \protect\footnotemark.}
  \label{fig:foto_max232}
\end{figure}

\footnotetext{Imagen tomada de \url{https://www.alldatasheet.com/datasheet-pdf/pdf/73074/MAXIM/MAX232.html}}   
     


\item ESP32-C3: microcontrolador de bajo consumo que actúa como unidad de procesamiento y comunicación. Integra conectividad y capacidad de comunicación serial, lo que lo hace adecuado para la interacción con el contador y con el módulo GPRS \cite{esp32c3IDF}.

En la figura \ref{fig:esp32} se muestra el microntrolador utilizado en campo.

\begin{figure}[H]
  \centering
  \includegraphics[width=0.30\linewidth]{./Figures/fotoEsp32c3.png}
  \caption{Microcontrolador ESP32-C3 utilizado en los nodos de campo \protect\footnotemark.}
  \label{fig:esp32}
\end{figure}

\footnotetext{Imagen tomada de \url{https://docs.espressif.com/projects/esp-idf/en/v5.0/esp32c3/hw-reference/esp32c3/user-guide-devkitm-1.html}}

\item Módulo GPRS SIM800L: permite la conexión de los dispositivos a la red celular \cite{sim800l_datasheet}, que asegura la transmisión de datos al servidor central mediante MQTT. Se eligió por su bajo costo, disponibilidad en el mercado y facilidad de integración con el ESP32-C3.

En la figura \ref{fig:foto_sim800l} se aprecia el módulo SIM800L que implementa la conectividad celular GPRS.

\begin{figure}[H]
  \centering
  \includegraphics[width=0.4\linewidth]{./Figures/fotoSim800l.png}
  \caption{Módulo SIM800L \protect\footnotemark.}
  \label{fig:foto_sim800l}
\end{figure}

\footnotetext{Imagen tomada de \url{https://www.alldatasheet.com/datasheet-pdf/pdf/1741389/SIMCOM/SIM800L.html}}

\item Servidor central: responsable de recibir los datos transmitidos, almacenarlos en una base de datos y responder a las solicitudes de los usuarios.

\item PC con navegador web: constituye el medio de interacción del usuario final con el sistema, a través de la interfaz desarrollada.

\item Fuente de alimentación: fuente principal con batería interna recargable y recarga mediante panel solar, capaz de cubrir los picos de consumo del SIM800L. Incluye regulador de tensión y capacitores que estabilizan el voltaje y evitan reinicios.

\item Carcasa y gabinete: protección para el contador y el módulo de comunicación (ESP32-C3 y SIM800L) en un gabinete cerrado resistente a humedad, polvo y vibraciones, con disipación térmica y reducción de interferencias electromagnéticas.

\item Componentes adicionales: filtros (capacitores) para garantizar estabilidad y evitar reinicios inesperados.


\end{itemize}


La integración de estos componentes asegura que el sistema pueda operar en condiciones reales y que cumpla con los requisitos.


\section{Tecnologías de software aplicadas}
El desarrollo del prototipo se sustenta en la integración de tecnologías de software abiertas y estandarizadas, seleccionadas por su solidez, amplia adopción en la comunidad tecnológica y capacidad para interoperar de manera eficiente entre los distintos componentes del sistema:

\begin{itemize}

\item MQTT con Eclipse Mosquitto. 
La mensajería ligera y bidireccional se implementa mediante el protocolo MQTT, operando sobre el broker Eclipse Mosquitto \cite{mosquitto}. Este software es ampliamente utilizado en entornos de Internet de las Cosas (IoT).

\item API REST con Node.js  y Express. 
El backend se desarrolla en Node.js \cite{nodejs}, un entorno de ejecución orientado a aplicaciones de red no bloqueantes, y se estructura con Express \cite{expressjs}, un framework ligero que facilita la creación de servicios RESTful.

\item Base de datos relacional MySQL. 
La persistencia de los datos se implementa en MySQL \cite{mysql}, un sistema gestor de bases de datos relacionales ampliamente adoptado. En ella se almacenan de manera estructurada tanto los eventos de tránsito como los estados reportados por los equipos.

\item ORM con Sequelize.
Para abstraer la interacción con la base de datos y mantener independencia frente a cambios en la capa de persistencia, se utiliza Sequelize \cite{sequelize}, un ORM para Node.js que permite definir modelos y relaciones de manera declarativa.  

\item Sistema de logging con Winston.
La trazabilidad de eventos y errores se gestiona mediante Winston \cite{winston}, una biblioteca de logging que soporta múltiples transportes y permite configurar niveles de severidad, formatos y timestamps.  

\item Middleware de registro HTTP con Morgan.
Para capturar y auditar el tráfico entrante al backend se emplea Morgan \cite{morgan}, un middleware especializado en logging de peticiones HTTP, que complementa funcionalidad de Winston.  

\item Interfaz web con Ionic y Angular.
Para la interacción con el usuario final se diseñó una aplicación accesible desde cualquier navegador, construida con Ionic \cite{ionic} y Angular \cite{angular}. La primera aporta componentes visuales responsivos que aseguran usabilidad tanto en entornos de escritorio como en dispositivos móviles.

\item Orquestación con Docker Compose.
Para simplificar el despliegue y la gestión de los servicios del backend, se utiliza Docker Compose \cite{docker_compose}. Esto permite levantar de manera consistente los contenedores de la API REST, el broker MQTT y la base de datos MySQL, que asegura que todas las dependencias se inicien en el orden correcto y facilita la replicación del entorno en desarrollo, prueba y producción.

\item  Firmware para ESP32-C3 con ESP-IDF.
El microcontrolador ejecuta un firmware desarrollado en C/C++, el cual utiliza el framework oficial de ESP-IDF \cite{espidf}, el framework oficial de Espressif. Este entorno proporciona bibliotecas optimizadas. El firmware controla la captura de datos desde el contador mediante RS-232, gestiona la comunicación con el módulo SIM800L mediante comandos AT y establece la conexión con el broker Mosquitto a través de MQTT.

\end{itemize}


\section{Software de control de versiones}
Para la gestión del código fuente se empleó GitHub \cite{github}, una plataforma que se basa en el sistema de control de versiones Git. Esta herramienta permitió almacenar el repositorio central de manera segura, registrar el historial de cambios y garantizar la trazabilidad de cada modificación realizada durante la etapa de desarrollo del prototipo.

El repositorio incluye el firmware del ESP32-C3, el backend en Node.js con Express y la interfaz web desarrollada con Ionic y Angular. Esto facilita mantener el código organizado, con la posibilidad de crear ramas independientes para implementar nuevas funciones y, posteriormente, integrarlas al código principal mediante pull requests, lo que permite revisar y validar los cambios antes de su incorporación definitiva.

Asimismo, se emplean issues para documentar incidencias, planificar tareas y realizar el seguimiento de los avances. Esta práctica favorece la organización del trabajo y permite mantener un registro claro de los problemas detectados y las decisiones adoptadas.



\chapter{Diseño e implementación} % Main chapter title

\label{Chapter3}

En este capítulo se describe la arquitectura global del prototipo, se detalla cada módulo de hardware y software que lo compone, y se documentan las decisiones de implementación y los criterios de diseño. Se explican los flujos de datos entre el dispositivo de campo, el broker MQTT, el backend (API REST), la interfaz web, se resumen las consideraciones para el despliegue y el monitoreo post-implantación.


\section{Arquitectura del sistema}

La arquitectura propuesta separa de forma explícita el dispositivo de campo (contador + ESP32-C3 + SIM800L), el transporte de mensajes (broker MQTT) y los servicios de aplicación (API REST, persistencia y frontend). Esta separación facilita la interoperabilidad y permite desplegar la solución de forma local, remota o híbrida según las políticas institucionales


\subsection{Descripción ampliada de bloques y responsabilidades}

El sistema se organiza en cinco bloques principales, cada uno con responsabilidades claramente definidas para garantizar un flujo de datos confiable, eficiente y seguro. Cada bloque cumple funciones específicas dentro del ciclo de captura, transmisión, procesamiento y visualización de los eventos, que asegura trazabilidad, persistencia y control de comandos remotos. Se describen los bloques y sus responsabilidades:


\begin{itemize}

  \item {Dispositivo de campo:} el nodo de campo integra el contador existente (salida RS-232), un microcontrolador ESP32-C3 y un módem GPRS SIM800L. El firmware, desarrollado sobre \texttt{ESP-IDF}, realiza las siguientes funciones: lectura continua de la trama serial, parsing tolerante a ruido, preprocesado (validación, normalización de campos y asignación de sello temporal UTC), encolamiento FIFO de eventos, gestión de reintentos y publicación MQTT cuando hay conectividad. Además, el nodo se suscribe a los tópicos de comandos y publica telemetría y acks. En el nodo se implementa persistencia mínima (registro de comandos pendientes y últimas N tramas) para recuperación tras reinicio.

  \item {Transporte (broker MQTT):} el broker actúa como bus de mensajes desacoplado. Se recomienda emplear Eclipse Mosquitto en la etapa inicial y evaluar brokers gestionados para despliegues a mayor escala. El broker gestiona autenticación por credenciales, control de tópicos y cifrado. Se emplean tópicos jerárquicos por dispositivo para facilitar filtrado y autorización: 

\begin{itemize}
  \item \texttt{dispositivo/\{id\}/medicion}
  \item \texttt{dispositivo/\{id\}/comando} 
  \item \texttt{dispositivo/\{id\}/respuesta}
\end{itemize}


\item {Servidor central:} el servidor central reúne dos responsabilidades principales: 

\begin{itemize}  

\item  Componente suscriptor MQTT que valida, transforma y enruta mensajes hacia la lógica de negocio y la persistencia.

\item API REST  ofrece servicios de consulta, gestión y comandos, manteniendo independencia del broker para permitir nuevos consumidores. Los datos se almacenan en MySQL con soporte para rangos temporales, alto rendimiento y auditoría de comandos.
 \end{itemize}

\item Cliente/Visualización: La interfaz web, desarrollada en Ionic + Angular, consume la API REST para consultas históricas y eventos en tiempo real. Ofrece visualización de eventos, consultas filtradas, envío de comandos y un panel de telemetría para mantenimiento.
\end{itemize}

Se separó el broker de la aplicación, se usó MQTT por su eficiencia y se creó una API REST en Node.js para gestión y autenticación.

La figura \ref{fig:diag_arquitectura} muestra el diagrama de arquitectura del sistema y el flujo de datos.


\begin{figure}[H]
  \centering
  \includegraphics[width=0.45\linewidth]{./Figures/diagArq.png}
  \caption{Diagrama de arquitectura del sistema y el flujo de datos.}
  \label{fig:diag_arquitectura}
\end{figure}



\subsection{Flujo de datos} 
El flujo de datos del sistema describe cómo se captura, procesa y comunica la información desde el contador hasta la interfaz de usuario, lo que permite la trazabilidad, persistencia y control de los eventos y comandos. Se detallan las etapas principales:

\begin{itemize}

  \item Detección: el contador detecta un paso, acumula y en determinado intervalo envía una trama por RS-232 al ESP32-C3.

  \item Preprocesado en nodo: el firmware valida la trama, añade sello temporal y metadatos, y encola el evento en memoria (FIFO).

  \item Transmisión: cuando la conexión GPRS está disponible, el nodo publica las mediciones en el tópico MQTT \texttt{dispositivo/{id}/medicion}.
  
  \item Ingesta y persistencia: el broker Mosquitto entrega el mensaje al suscriptor backend, el servicio valida el payload y persiste el registro en la base de datos MySQL.
  
  \item Visualización/Control: la interfaz web consulta la API REST para datos históricos y recibe notificaciones en tiempo real.
  
  \item Emisión de comandos (desde UI): el operador genera un comando en la interfaz, la UI envía
  \texttt{POST /app/comando} al backend, que crea un \texttt{comm\_id} único y publica en \texttt{dispositivo/{id}/comando}.
  
  \item Recepción nodo/Entrega al contador: el ESP32-C3 en \\ \texttt{dispositivo/{id}/comando} recibe el comando, valida \texttt{cmd\_id} y lo envía al contador por RS-232, se aplica un timeout configurable por comando.
  
  \item Ejecución y ack: el contador ejecuta la orden y responde por RS-232, el firmware publica el ack/resultado en \texttt{dispositivo/{id}/respuesta} con \texttt{cmd\_id} y \texttt{status} (\texttt{ok}, \texttt{failed}, \texttt{timeout}, \texttt{value}).

  \item Actualización en backend y UI: el suscriptor MQTT del backend recibe el ack, actualiza la tabla \texttt{respuesta} (campo \texttt{valor}, \texttt{ack\_ts}) y notifica a la UI para que el operador vea el resultado.
\end{itemize}
\pagebreak

La figura \ref{fig:diag_secuencia} muestra el diagrama de secuencia del flujo de datos.

\begin{figure}[H]
  \centering
  \includegraphics[width=1.15\linewidth]{./Figures/diagSecuencia.png}
  \caption{Diagrama de secuencia del flujo de datos.}
  \label{fig:diag_secuencia}
\end{figure}



\section{Arquitectura del nodo}
La arquitectura de cada nodo se diseñó con el objetivo de reutilizar los contadores de tránsito actualmente desplegados en las rutas nacionales. Cada nodo integra varios módulos que permiten la adquisición, procesamiento y transmisión de los datos de tránsito:

\begin{itemize}
    \item Contador de tránsito modelo DTEC: dispositivo encargado de detectar el paso de vehículos y generar tramas de datos con la información del evento.
   
    \item Módulo de adaptación RS-232/TTL (MAX232): circuito de conversión de niveles eléctricos que asegura compatibilidad entre la interfaz serial del contador (RS-232) y el microcontrolador (niveles TTL).
      
    \item ESP32-C3: microcontrolador que ejecuta el firmware desarrollado sobre ESP-IDF. Sus funciones incluyen el preprocesamiento de eventos, el encolamiento FIFO, la suscripción a comandos remotos y la gestión integral de la comunicación con el servidor.
   
    \item Módulo de comunicación GPRS (SIM800L): interfaz de conectividad celular que publica eventos en el broker MQTT, recibe comandos desde el servidor y retransmite respuestas o estados del nodo.
    



%
%\subsection{ESP32-C3 (unidad de control)}
%El ESP32-C3 constituye el núcleo de procesamiento del nodo de campo. Se trata de un microcontrolador de bajo consumo con conectividad, elegido principalmente por su capacidad de cómputo, su soporte de entornos de desarrollo abiertos y la disponibilidad de bibliotecas optimizadas para protocolos de comunicación.
%
%\begin{itemize}
%
%\item Rol en la arquitectura: coordina la recepción de eventos desde el contador a través de RS-232, gestiona el encolamiento FIFO, controla la comunicación con el módem SIM800L mediante comandos AT y actúa como cliente MQTT.
%
%\item Entorno de desarrollo: el firmware se desarrolló sobre ESP-IDF, el framework oficial de Espressif, que permite gestionar tareas concurrentes mediante FreeRTOS y facilita la integración de librerías de red y drivers UART.
%
%\item Ventajas técnicas: bajo costo, consumo reducido, capacidad de ejecutar varias tareas en paralelo y soporte nativo para criptografía y seguridad en comunicaciones.
%
%\item Consideraciones de diseño: se provee una correcta disipación térmica y  estabilidad de la alimentación, especialmente durante la transmisión de datos.
%
%\end{itemize}
%
%\subsection{Módulo GPRS SIM800L}
%
%El módulo SIM800L implementa la conectividad celular GPRS, que permite la transmisión bidireccional de datos entre dispositivos remotos y servidor centralizado.
%
%
%\begin{itemize}
%
%\item Funciones principales: establecimiento de sesiones TCP/IP sobre GPRS, controlado por el ESP32-C3 mediante comandos AT. Soporta publicación y suscripción MQTT a través de sockets TCP persistentes.
%
%\item Integración con ESP32-C3: la comunicación entre ambos se realiza mediante UART secundaria. El firmware implementa comandos de inicialización, registro en red, apertura de contexto y gestión de reconexiones.
%
%\item Aspectos críticos: el SIM800L presenta picos de consumo que pueden superar los 2 A durante la transmisión, se  dispone de una fuente con suficiente margen y filtros adecuados para evitar reinicios inesperados.
%
%\item Limitaciones: ancho de banda reducido (máximo teórico de 85,6 kbps en GPRS), lo que refuerza la decisión de emplear MQTT por su bajo overhead.
%
%\end{itemize}
%
%
%\subsection{Contador de tránsito y comunicación RS-232}
%
%El sistema de conteo existente genera tramas con información de los eventos de paso de vehículos acumulados en un intervalo de tiempo (por defecto 1 hora). La comunicación con el ESP32-C3 se establece mediante interfaz RS-232, estándar ampliamente utilizado para transmisión serial de datos.
%
%\begin{itemize}
%\item Formato de trama:  el equipo incluye identificador de Dipositivo, timestamp local, valores de clasificación tránsito por carril.
%
%\item Adaptación eléctrica: se utiliza un conversor de niveles (MAX232) para adaptar las señales RS-232 al rango TTL del microcontrolador.
%
%\item Ventaja: la reutilización de la interfaz serial del contador evita modificar el sistema de detección existente, reduciendo costos de integración.
%
%
%\end{itemize}
%
%
%En la figura \ref{fig:foto_dtec} se observa el contador de tránsito. Este dispositivo constituye la base del sistema de detección de eventos y se mantiene sin modificaciones en su lógica interna. 
%
%
%\begin{figure}[htbp]
%  \centering
%  \includegraphics[width=0.5\linewidth]{./Figures/fotoDTEC.jpeg}
%  \caption{Contador de tránsito DTEC \protect\footnotemark.}
%  \label{fig:foto_dtec}
%\end{figure}
%
%\footnotetext{Imagen tomada de \url{http://transito.vialidad.gob.ar/}}

\item Alimentación y montaje: para garantizar el funcionamiento continuo del nodo en entornos de campo,
se deben considerar dos aspectos fundamentales:
\begin{itemize}
\item Fuente de alimentación: se implementó un sistema de energía basado en una batería interna recargable, específicamente dimensionada para cubrir los picos de consumo del módulo SIM800L durante las transmisiones de datos. La autonomía del sistema está garantizada mediante un panel solar que mantiene la carga de la batería de forma continua.

Para estabilizar la entrega de energía se incorporó un módulo regulador de tensión que garantiza el nivel adecuado de voltaje para el módem. El circuito se complementó con capacitores dimensionados para absorber los picos de tensión del SIM800L, evitando caídas de voltaje que puedan reiniciar el sistema.

La figura \ref{fig:diag_conexiones} se presenta el diagrama de conexión entre los módulos del sistema, detallando las líneas de comunicación serie RS232, la interfaz UART entre el microcontrolador y el módem GSM, así como la distribución de la alimentación eléctrica y los circuitos de estabilización de potencia.

\begin{figure}[H]
  \raggedleft
  \includegraphics[width=0.8\linewidth]{./Figures/diagConexion.png}
  \caption{Diagrama de conexión entre los módulos del sistema.}
  \label{fig:diag_conexiones}
\end{figure}



\item Carcasa y gabinete: tanto el contador como el módulo de comunicación remota (ESP32-C3 y SIM800L) cuentan con su propia carcasa de protección y, adicionalmente, ambos se alojan en un gabinete metálico estandarizado ya existente en la infraestructura vial, diseñado para resistir humedad, polvo y vibraciones. Esta solución aprovecha la protección ambiental, disipación térmica y blindaje electromagnético del gabinete original, que garantiza la confiabilidad del sistema en condiciones de intemperie.

En la figura \ref{fig:foto_gabinete} se observa el contador de tránsito instalado en campo, junto con su gabinete de protección y la batería interna que asegura autonomía energética. El conjunto se encuentra montado al costado de la ruta, en condiciones reales de operación, lo que permite apreciar el encapsulado diseñado para resistir las exigencias ambientales del entorno vial.


\begin{figure}[H]
  \centering
  \includegraphics[width=1\linewidth]{./Figures/fotoGabinete.jpeg}
  \caption{Fotografia contador de tránsito DTEC instalado en campo \protect\footnotemark.}
  \label{fig:foto_gabinete}
\end{figure}

\end{itemize}
\footnotetext{Imagen tomada de \url{http://transito.vialidad.gob.ar/}}

\end{itemize}

\section{Desarrollo del backend}

El backend constituye el núcleo lógico del sistema, encargado de articular la comunicación entre los dispositivos de campo, la base de datos y la interfaz de usuario. Su diseño se basó en principios de modularidad, escalabilidad y seguridad, empleando un conjunto de tecnologías ampliamente utilizadas en entornos de Internet de las Cosas (IoT).


\subsection{Arquitectura y tecnologías}

El servicio se implementó en Node.js con Express, organizando la aplicación en controladores, rutas y middlewares, y usando Sequelize \cite{sequelize}, un ORM \cite{fowler2002patterns} que facilita los modelos y asegura independencia de la persistencia.

La comunicación con los dispositivos se realiza mediante tópicos MQTT en Eclipse Mosquitto. Las mediciones se publican en \texttt{dispositivo/{id}/medicion} y el backend las valida y guarda en MySQL, mientras que los comandos y respuestas se gestionan en \texttt{dispositivo/{id}/comando} y \texttt{dispositivo/{id}/respuesta}. El despliegue usa Docker Compose \cite{docker_compose} para ejecutar backend, base de datos y broker \cite{mqttSpec} en contenedores, y el registro se maneja con Winston \cite{winston} y Morgan \cite{morgan} para trazabilidad completa.

En la figura \ref{fig:diagrama_backend} se observa el diagrama de flujo de información del backend.


\begin{figure}[H]
 
  \centering
  \includegraphics[width=0.9\linewidth]{./Figures/diagFlujoConexionBackend.png}
  \captionof{figure}{Diagrama de flujo de información del Backend.}
  \label{fig:diagrama_backend}
  \end{figure}

\FloatBarrier
\subsection{Funcionalidades principales}
El sistema cuenta con varias funcionalidades esenciales que permiten gestionar de manera eficiente los dispositivos, los eventos generados por ellos, los comandos enviados y la seguridad de acceso. Estas funcionalidades se detallan a continuación:

\begin{itemize}
    \item Gestión de dispositivos: alta, baja, modificación y consulta.
    \item Gestión de eventos: almacenamiento de detecciones y consultas filtradas por dispositivo o rango temporal.
    \item Gestión de comandos: emisión de órdenes a un dispositivo, persistencia de la orden con identificador único (cmd\_id) y actualización según respuesta.
    \item Estado de dispositivos: consulta de parámetros como nivel de batería, temperatura o conectividad.
    \item Autenticación y autorización: control de acceso mediante tokens JWT.
\end{itemize}


\subsection{Organización en controladores}

La lógica de negocio del backend se organiza en controladores, cada uno asociado a un recurso del sistema, lo que asegura una clara separación de responsabilidades, facilita el mantenimiento y permite la escalabilidad de la aplicación. Entre los controladores principales se encuentran los siguientes:

\begin{itemize}
  \item DispositivoController: gestiona las operaciones CRUD sobre los dispositivos de campo, además de registrar los eventos recibidos vía MQTT y asociarlos a un dispositivo específico.
  \item MedicionController: encapsula la lógica de ingesta de eventos de tránsito, validación de payloads y persistencia en la base de datos.
  \item ComandoController: administra la emisión y seguimiento de comandos remotos, generando un \texttt{cmd\_id} único y actualizando el estado conforme se reciben los \texttt{ack}.
  \item RespuestaController: centraliza la recepción de estados y telemetría (batería, conectividad), que garantiza que la base de datos refleje la situación en tiempo real.
  \item UserController: implementa el ciclo de vida de usuarios y la autenticación mediante JWT\cite{jwtRFC7519}, así como la validación de permisos en cada endpoint.
\end{itemize}

En la figura \ref{fig:diagrama_controladores} se observa el diagrama con la disposición de los controladores y flujo de dependencias.

\begin{figure}[H]
 
  \centering
  \includegraphics[scale=0.11]{./Figures/diagDispoControlladores.png}
  \captionof{figure}{Diagrama con la disposición de los controladores y flujo de dependencias.}
  \label{fig:diagrama_controladores}
  \end{figure}



\subsection{Mapa de endpoints}

El backend expone una serie de endpoints REST que conforman la interfaz principal de comunicación con los servicios de aplicación y los dispositivos de campo.  
A continuación, se presenta la tabla general de los endpoints y controladores mas importantes para el flujo:

\begin{table}[H]
	\centering
	\caption[Endpoints REST principales]{Endpoints REST principales expuestos por el backend, junto con el controlador que implementa su lógica.}
	\begin{tabular}{l l p{3.5cm}}    
		\toprule
		\textbf{Endpoint} & \textbf{Controlador} & \textbf{Descripción} \\
		\midrule
		GET /dispositivo & DispositivoController & Lista todos los dispositivos registrados \\
		GET /dispositivo/\{id\} & DispositivoController & Devuelve información de un dispositivo específico \\
		POST /dispositivo & DispositivoController & Alta de un nuevo dispositivo \\
		PATCH /dispositivo/\{id\} & DispositivoController & Actualización de atributos de un dispositivo \\
		DELETE /dispositivo/\{id\} & DispositivoController & Eliminación de un dispositivo \\
		\addlinespace
		POST /medicion & MedicionController & Crea  mediciones de un dispositivo \\
		GET /medicion/dispositivo/\{id\} & MedicionController & Consultar mediciones por dispositivo \\
		GET /medicion/range & MedicionController & Consultar mediciones por rango temporal \\
		\addlinespace
		POST /comando & ComandoController & Crear un comando remoto y publicarlo en MQTT \\
		GET /comando/\{id\} & ComandoController & Consultar un comando  \\
		\addlinespace
		GET /respuesta/\{id\} & RespuestaController & Consultar respuesta de un comando \\
		\addlinespace
		POST /usuario/login & UserController & Autenticación de usuario, devuelve token JWT \\
		POST /usuario & UserController & Alta de usuario \\
		GET /usuario & UserController & Listar usuarios registrados \\
		DELETE /usuario/\{id\} & UserController & Eliminar usuario \\
		\bottomrule
		\hline
	\end{tabular}
	\label{tab:endpoints}
\end{table}


\subsection{Seguridad y extensibilidad}

Además de la autenticación mediante JWT, todos los endpoints aplican validaciones y sanitización de parámetros de entrada y salida. El sistema de logging, implementado con Winston y Morgan, garantiza trazabilidad de las operaciones tanto en la capa HTTP como en la mensajería MQTT. La arquitectura modular basada en controladores permite extender el backend con nuevos recursos o funcionalidades sin afectar la lógica ya implementada.


\section{Desarrollo del frontend}

El frontend del sistema se diseñó como una \textit{Single Page Application} se desarrolla en Ionic con Angular y TypeScript.  
El objetivo es proporcionar una interfaz moderna e intuitiva, accesible desde navegador, que permita al operador autenticarse, supervisar en tiempo real los eventos captados por los contadores de tránsito, consultar históricos almacenados en la base de datos y emitir comandos remotos hacia los nodos de campo.  

\subsection{Arquitectura y tecnologías}

El cliente web se estructura en componentes reutilizables de Ionic, lo que facilita la navegación y asegura un diseño responsivo tanto en entornos de escritorio como móviles.  
La comunicación con el backend se realiza mediante peticiones HTTP a la API REST, y en casos donde se requiere actualización en tiempo real se emplea un canal de notificación basado en WebSocket.


\subsection{Funcionalidades principales}

El frontend integra las siguientes funciones clave:
\begin{itemize}
    \item Login de usuario: ingreso con credenciales, validación contra la API y obtención de un token JWT.
    \item Listado de dispositivos: muestra todos los contadores registrados, junto con información de ubicación y estado básico.
    \item Detalle de dispositivo: despliega datos específicos de un contador y últimas tramas recibidas.
    \item Panel de mediciones: permite visualizar los eventos de tránsito procesados, con actualización dinámica cuando el dispositivo transmite nuevas tramas.
    \item Historial de eventos: consulta de registros almacenados en la base de datos, filtrados por dispositivo y rango temporal.
    \item Envío de comandos: panel que permite emitir órdenes remotas hacia el nodo de campo, como reset del contador, modificación u obtención de parámetros. El sistema verifica el acuse de recibo de cada orden y muestra al usuario el resultado correspondiente (ok, failed, timeout o value).
    
\end{itemize}

En la figura \ref{fig:diagrama_controladores} se observa la estructura de los componentes por pantalla.

\begin{figure}[H]
 
  \centering
  \includegraphics[width=1\linewidth]{./Figures/diagEstructuraComponentesPantallasFrontend.png}
  \captionof{figure}{Diagrama de estructura de los componentes por pantalla.}
  \label{fig:diagrama_controladores}
\end{figure}



\subsection{Integración con el backend}

Todas las operaciones del frontend se apoyan en los endpoints REST definidos en el backend (ver Sección~\ref{tab:endpoints}).  
Cada petición incluye en sus cabeceras el token JWT obtenido en el login, lo que garantiza que solo usuarios autorizados puedan acceder a datos sensibles o emitir comandos.  
El backend devuelve respuestas en formato JSON, que son interpretadas y representadas en la interfaz en tiempo real, lo que asegura consistencia entre la vista del operador y el estado real de los dispositivos.

En la figura \ref{fig:diagramaflujorestapi} se observa el diagrama de flujo de comunicación con el backend.

\begin{figure}[H]
 
  \centering
  \includegraphics[width=1.1\linewidth]{./Figures/diagFlujoRestApi.png}
  \captionof{figure}{Diagrama de flujo de comunicación con el backend.}
  \label{fig:diagramaflujorestapi}
\end{figure}


\clearpage

\section{Despliegue del sistema}

El despliegue del sistema comprende la puesta en marcha coordinada de los distintos servicios que componen la arquitectura: el broker MQTT, la API REST, la base de datos relacional y la interfaz web.  
El objetivo es trasladar el prototipo desde un entorno de desarrollo hacia un entorno productivo, que asegura escalabilidad, confiabilidad y capacidad de monitoreo post-implantación.  

\subsection{Entorno productivo y configuración}

Para simplificar la orquestación se emplea Docker Compose, lo que permite instanciar todos los servicios en contenedores aislados pero comunicados entre sí.  
Cada componente cumple un rol específico:  

\begin{itemize}
    \item Broker MQTT (Eclipse Mosquitto): configurado como servicio de mensajería central. Se habilitan credenciales de acceso, control de tópicos por dispositivo y soporte de cifrado TLS en despliegues productivos. Su rol es recibir eventos desde los nodos de campo y distribuirlos a los suscriptores autorizados (API REST u otros consumidores).
    
    \item API REST (Node.js/Express): implementada como contenedor independiente, integra lógica de negocio y suscripción al broker MQTT. De esta forma, cada evento recibido es validado, transformado y almacenado en la base de datos. La API expone endpoints para:
    \begin{itemize}
        \item Gestión de dispositivos y usuarios.
        \item Consulta de eventos por rango temporal o por dispositivo.
        \item Emisión y seguimiento de comandos remotos.
        \item Acceso al estado operativo y telemetría de cada nodo.
    \end{itemize}
    Todos los endpoints están protegidos mediante autenticación con tokens JWT.
    
    \item Base de datos relacional (MySQL): instancia dedicada a persistencia de información. Su esquema incluye tablas de dispositivos, eventos, comandos y usuarios. Se definen índices para optimizar consultas históricas y se configuran respaldos automáticos diarios.
    
    \item Interfaz web (Ionic/Angular): desplegada como servicio accesible en navegador. Consume la API REST para mostrar eventos en tiempo real, ejecutar comandos y consultar históricos.
\end{itemize}

\subsection{Monitoreo post-implantación}

Una vez desplegado el sistema, resulta fundamental contar con mecanismos de monitoreo que permitan evaluar su correcto funcionamiento en campo:  

\begin{itemize}
    \item Logs centralizados: tanto el backend como el broker MQTT registran eventos en archivos y consola. Se  integra con Grafana para correlacionar métricas.
    
    \item Alertas y métricas: mediante Grafana es posible recolectar indicadores de CPU, memoria y estado de contenedores. También se pueden graficar métricas de tráfico MQTT (mensajes publicados, latencias, pérdidas).
    
    \item Supervisión de dispositivos: la API REST expone endpoints que informan conectividad y parámetros básicos (nivel de batería, último evento recibido). Estos datos se representan en la interfaz web como panel de salud del sistema.
    
    \item Respaldo y recuperación: la base de datos implementa backups automáticos y permite restauraciones parciales. Esto garantiza que el historial de eventos no se pierda ante fallas de hardware o corrupción de datos.
\end{itemize}



\section{Integración con la infraestructura existente}
Una de las principales ventajas de este diseño es que no requiere modificaciones internas en el contador de tránsito. El nodo recibe los pulsos de detección mediante la interfaz RS-232, que preserva la integridad del equipo original. 

El ESP32-C3 no se limita a reenviar datos, sino que añade valor al sistema al realizar un preprocesado local: filtra tramas, agrupa eventos en función de ventanas de tiempo y asegura la transmisión con políticas de reintento. Asimismo, la conexión con el servidor central mediante MQTT garantiza interoperabilidad con aplicaciones externas y facilita la escalabilidad del sistema.

En este contexto, los nodos de campo cumplen un doble rol: por un lado, son captadores de datos provenientes de los sensores de tránsito y por otro, actúan como puntos de control remoto, capaces de ejecutar comandos enviados desde la plataforma central. Esta dualidad refuerza la flexibilidad del sistema y lo hace adaptable a distintas políticas de gestión vial.






\chapter{Ensayos y Resultados} % Main chapter title
\label{sec:ensayos-resultados}
En este capítulo se presentan en detalle los ensayos realizados sobre el sistema desarrollado, con el propósito de validar su funcionamiento en condiciones representativas de uso real. 
Los ensayos se organizaron en diferentes niveles: banco de pruebas en laboratorio, validación de la API REST, pruebas unitarias e integración de componentes, 
pruebas del frontend, prueba final de integración end-to-end y una comparación con soluciones comerciales y académicas.  


\section{Banco de pruebas}
\label{sec:banco-pruebas}

El banco de pruebas se diseñó con el propósito de reproducir las condiciones reales de operación del sistema de detección de tránsito, garantizando la validez de los resultados obtenidos en un entorno controlado. Este entorno permitió evaluar la robustez del firmware, la estabilidad de las comunicaciones y la capacidad del backend para procesar eventos en diferentes escenarios de conectividad.

El objetivo principal fue analizar el comportamiento integral del sistema ante situaciones representativas de campo, incluyendo la pérdida temporal del enlace GPRS, el almacenamiento local de eventos y la recuperación automática una vez restablecida la conexión.

\subsection{Diseño del entorno de pruebas}

El banco se compuso de los siguientes elementos principales:

\begin{itemize}
    \item Contador de tránsito DTEC: configurado para generar tramas de detección simuladas, con distintos intervalos de paso vehicular.
    \item Nodo de campo (ESP32-C3 + SIM800L): encargado de recibir las tramas RS-232, almacenarlas temporalmente y transmitirlas mediante MQTT al servidor central.
    \item Servidor de backend: implementado en Node.js/Express, con base de datos MySQL y broker Eclipse Mosquitto, desplegado mediante Docker Compose.
    \item Interfaz web de monitoreo: utilizada para visualizar en tiempo real los eventos recibidos y el estado de los dispositivos.
\end{itemize}

El montaje permitió reproducir tres escenarios de prueba diferenciados:

\begin{enumerate}
    \item Conectividad estable: transmisión continua sin pérdidas de enlace.
    \item Conectividad intermitente: simulación de cortes GPRS aleatorios, verificando la persistencia de los datos en la cola interna del nodo.
    \item Modo desconectado prolongado: interrupción total de red durante intervalos extensos, evaluando la capacidad del firmware para conservar eventos en memoria y transmitirlos una vez reconectado.
\end{enumerate}

\subsection{Metodología experimental}

Las pruebas se realizaron mediante la generación de tramas seriales controladas, representando detecciones vehiculares. Se implementó un módulo de simulación que permitió enviar secuencias de tramas RS-232 al ESP32-C3, registrando tanto los tiempos de procesamiento como la cantidad de eventos almacenados en la cola FIFO.

Para la simulación de pérdida de conectividad, se forzó el corte del enlace GPRS  al módulo SIM800L, y se verificó que los mensajes no enviados quedaran encolados localmente. Una vez restablecida la conexión, los eventos se publicaron en el tópico MQTT correspondiente en: 

\begin{itemize}
  \item \texttt{dispositivo/\{id\}/medicion}
  \item \texttt{dispositivo/\{id\}/respuesta}
\end{itemize}

El backend, por su parte, registró la llegada de los eventos en la base de datos MySQL, verificando integridad, timestamps y ausencia de duplicaciones.


\subsection{Resultados y observaciones}

Los resultados experimentales demostraron que el sistema fue capaz de:

\begin{itemize}
    \item Mantener la integridad de los datos bajo escenarios de conectividad inestable.
    \item Garantizar la entrega de eventos mediante la cola FIFO implementada en el firmware.
    \item Ejecutar comandos remotos y recibir respuestas de manera confiable.
    \item Reanudar correctamente la transmisión tras cortes de red sin pérdida de información.
\end{itemize}

En promedio, los tiempos de publicación por evento se mantuvieron dentro de rangos aceptables (entre 300 y 600 ms) en escenarios con conexión estable, y con demoras proporcionales en los períodos de reconexión.

%\begin{figure}[H]
%    \centering
%    \includegraphics[width=0.9\textwidth]{banco_pruebas.png}
%    \caption[Diagrama del proceso de pruebas]{Diagrama del proceso de pruebas utilizado para la validación del flujo de eventos y comandos en el sistema.}
%    \label{fig:banco-pruebas}
%\end{figure}

En conclusión, el banco de pruebas permitió validar la arquitectura propuesta, que permite confirmar un comportamiento confiable frente a condiciones reales de operación y que demuestra la efectividad de los mecanismos de encolado y retransmisión implementados.

\section{Pruebas de la API REST}
\label{sec:pruebas-api}

En esta sección se detallan las pruebas realizadas sobre la API REST que se implemento en el backend del sistema. El propósito de estas pruebas fue validar la correcta interacción entre los distintos componentes (backend, base de datos y broker MQTT) y comprobar la integridad, seguridad y rendimiento de las operaciones ofrecidas por los endpoints definidos.  




\subsection{Objetivos y alcance}

\subsection{Diseño del entorno de pruebas}

El banco se compuso de los siguientes elementos principales:

\begin{itemize}
    \item Contador de tránsito DTEC: configurado para generar tramas de detección simuladas, con distintos intervalos de paso vehicular.
    \item Nodo de campo (ESP32-C3 + SIM800L): encargado de recibir las tramas RS-232, almacenarlas temporalmente y transmitirlas mediante MQTT al servidor central.
    \item Servidor de backend: implementado en Node.js/Express, con base de datos MySQL y broker Eclipse Mosquitto, desplegado mediante Docker Compose.
    \item Interfaz web de monitoreo: utilizada para visualizar en tiempo real los eventos recibidos y el estado de los dispositivos.
\end{itemize}


El montaje permitió reproducir tres escenarios de prueba diferenciados:

\begin{enumerate}
    \item Conectividad estable: transmisión continua sin pérdidas de enlace.
    \item Conectividad intermitente: simulación de cortes GPRS aleatorios, verificando la persistencia de los datos en la cola interna del nodo.
    \item Modo desconectado prolongado: interrupción total de red durante intervalos extensos, evaluando la capacidad del firmware para conservar eventos en memoria y transmitirlos una vez reconectado.
\end{enumerate}



\begin{itemize}
    \item Diseño de escenarios de prueba: cada caso fue descrito en términos de entradas, condiciones ambientales (conectividad, ruido eléctrico, interrupciones), criterios de éxito y métricas a evaluar.
    \item \textbf{Instrumentación}: se utilizaron herramientas como \textit{Postman} para la API REST, \textit{Wireshark} para el análisis de tráfico MQTT, \textit{Mosquitto\_sub/pub} para validación manual de tópicos y \textit{Lighthouse} para pruebas de frontend.
    \item \textbf{Registro}: se configuró el backend con \texttt{Winston} y \texttt{Morgan} para guardar trazas en disco y en consola. Adicionalmente, se recolectaron métricas con scripts Python que midieron tiempos de respuesta y pérdidas de mensajes.
    \item \textbf{Criterios de aceptación}: se establecieron como condiciones mínimas de validación: 
    (i) latencia promedio menor a 500 ms en API REST, 
    (ii) pérdida de eventos inferior al 0.1\% en escenarios de conectividad intermitente,
    (iii) tiempo de recuperación ante desconexión GPRS menor a 30 segundos.
\end{itemize}

\section{Banco de pruebas}
\label{sec:banco-pruebas}

El banco de pruebas se diseñó para simular condiciones reales de operación de los contadores de tránsito, 
incluyendo conectividad GPRS intermitente y generación de eventos artificiales.  

%\begin{figure}[H]
%    \centering
%    \includegraphics[width=0.8\textwidth]{banco_pruebas.png}
%    \caption{Esquema del banco de pruebas con contador, ESP32-C3, %módulo SIM800L, broker MQTT y servidor backend.}
%    \label{fig:banco-pruebas}
%\end{figure}

Los objetivos principales fueron:  
\begin{itemize}
    \item Validar la correcta recepción de tramas RS-232 desde el contador.
    \item Evaluar el desempeño del firmware en el manejo de colas FIFO en memoria.
    \item Verificar la transmisión y reintento de eventos a través del protocolo MQTT.
    \item Confirmar la recepción de comandos desde el servidor y la emisión de respuestas de tipo \textit{acknowledge}.
\end{itemize}

Durante las pruebas se generaron tramas simuladas de detección y se forzaron desconexiones en el enlace GPRS. 
Se verificó que el firmware almacenaba los eventos en cola y los publicaba correctamente al restablecerse la conexión. 
También se probaron diferentes niveles de QoS en MQTT para medir la confiabilidad del envío.  

%\begin{table}[H]
%    \centering
%    \caption[Resultados de pruebas MQTT]{Resultados obtenidos en %pruebas de MQTT bajo distintos niveles de QoS y condiciones de %conectividad.}
%    \begin{tabular}{l c c c}
%    \toprule
%    \textbf{Escenario} & \textbf{QoS} & \textbf{Latencia media (ms)} %& \textbf{Pérdida de mensajes} \\
%    \midrule
%    Conectividad estable & 0 & 120 & 0.5\% \\
%    Conectividad estable & 1 & 180 & 0.0\% \\
%    Conectividad intermitente & 0 & 350 & 2.1\% \\
%    Conectividad intermitente & 1 & 420 & 0.1\% \\
%    Conectividad intermitente & 2 & 470 & 0.0\% \\
%    \bottomrule
%    \end{tabular}
%    \label{tab:pruebas-mqtt}
%\end{table}

Los resultados mostraron que el sistema pudo mantener la integridad de los eventos y ejecutar comandos remotos de forma confiable, incluso bajo condiciones adversas.  

\section{Pruebas de la API REST}
\label{sec:pruebas-api}

Las pruebas de la API REST se centraron en la validación de los endpoints implementados. 
Se emplearon colecciones de \textit{Postman} que automatizaron las consultas, con aserciones sobre códigos de estado y estructura de respuestas.  

%\begin{figure}[H]
%    \centering
 %   \includegraphics[width=0.75\textwidth]{postman_tests.png}
 %   \caption{Colección de pruebas en Postman utilizada para validar %los endpoints REST.}
%    \label{fig:postman}
%\end{figure}

\paragraph{Ejemplo de request y response para creación de dispositivo:}
\begin{verbatim}
POST /devices
{
  "nombre": "Nodo Ruta 9",
  "ubicacion": "Peaje km 255",
  "tipo": "contador"
}

Response 201:
{
  "message": "Dispositivo creado exitosamente",
  "id": 5
}
\end{verbatim}

Se verificaron:  
\begin{itemize}
    \item Operaciones CRUD sobre dispositivos y eventos.
    \item Autenticación y autorización mediante tokens JWT.
    \item Integración con el broker MQTT para el envío de comandos y registro de respuestas.
    \item Manejo de errores ante parámetros inválidos o solicitudes no autorizadas.
\end{itemize}

%\begin{table}[H]
%    \centering
%    \caption[Resultados de API REST]{Resultados promedio de latencia % en API REST durante las pruebas.}
%    \begin{tabular}{l c c}
%    \toprule
%    \textbf{Operación} & \textbf{Latencia media (ms)} & \textbf{Tasa %de error} \\
%    \midrule
%    Alta de dispositivo & 180 & 0.0\% \\
%    Consulta de dispositivos & 150 & 0.0\% \\
%    Creación de evento & 200 & 0.1\% \\
%    Consulta de eventos por rango & 230 & 0.0\% \\
%    Autenticación JWT & 170 & 0.0\% \\
%    \bottomrule
%    \end{tabular}
%    \label{tab:pruebas-api}
%\end{table}

Los resultados confirmaron que la API respondió en tiempos adecuados, con latencias promedio menores a 200 ms en pruebas locales y 350 ms en escenarios con GPRS.  

\section{Pruebas de componentes}
\label{sec:pruebas-componentes}

Se realizaron pruebas unitarias e integración sobre cada módulo del sistema:  

\begin{itemize}
    \item \textbf{Firmware}: validación del parsing de tramas RS-232, manejo de colas FIFO y reconexión GPRS.
    \item \textbf{API REST}: validación de controladores, middlewares y sanitización de datos.
    \item \textbf{Interfaz web}: verificación de consultas REST, visualización en tiempo real y envío de comandos.
    \item \textbf{Broker MQTT}: simulación de desconexiones para validar reintentos y niveles de QoS.
    \item \textbf{Manejo de errores}: pruebas forzadas de pérdida de conectividad y respuestas inválidas.
\end{itemize}

%\begin{figure}[H]
%    \centering
%    \includegraphics[width=0.7\textwidth]{tests_components.png}
%    \caption{Diagrama del flujo de pruebas de integración de %componentes.}
%    \label{fig:pruebas-componentes}
%\end{figure}

\section{Pruebas del frontend}
\label{sec:pruebas-frontend}

El frontend fue evaluado en términos de compatibilidad, rendimiento y usabilidad. 
Se verificó el funcionamiento en navegadores modernos (Chrome, Firefox, Edge) 
y en dispositivos móviles.  

Las métricas incluyeron:  
\begin{itemize}
    \item Tiempo de carga inicial (medido con Lighthouse).
    \item Latencia en consultas a la API.
    \item Velocidad de actualización de gráficos en tiempo real.
\end{itemize}

%\begin{table}[H]
%    \centering
%    \caption[Pruebas de frontend]{Métricas de frontend en distintos navegadores.}
%    \begin{tabular}{l c c c}
%    \toprule
%    \textbf{Navegador} & \textbf{Tiempo de carga (s)} & \textbf{FPS %gráficos} & \textbf{Compatibilidad} \\
%    \midrule
%    Chrome & 2.3 & 60 & Completa \\
%    Firefox & 2.7 & 55 & Completa \\
%    Edge & 2.5 & 58 & Completa \\
%    Móvil (Android) & 3.8 & 48 & Parcial (menú) \\
%    \bottomrule
%    \end{tabular}
%    \label{tab:pruebas-frontend}
%\end{table}

\section{Prueba final de integración}
\label{sec:prueba-integracion}

La prueba final consistió en validar el flujo completo: desde la detección de un vehículo 
en el contador hasta la visualización del evento en la interfaz web y la emisión de un comando remoto.  

%\begin{figure}[H]
%    \centering
%    \includegraphics[width=0.75\textwidth]{end_to_end.png}
%    \caption{Flujo end-to-end desde el nodo de campo hasta la interfaz web.}
%    \label{fig:end-to-end}
%\end{figure}

Los tiempos de ida y vuelta de un comando (round-trip) estuvieron entre 3 y 5 segundos 
en condiciones de red estables, aumentando a 12 segundos con conectividad intermitente.  

\section{Comparación con otras soluciones}
\label{sec:comparacion}

Finalmente, se realizó una comparación entre la solución desarrollada y otras alternativas 
comerciales y académicas, considerando criterios como costo, flexibilidad, escalabilidad 
y adecuación a entornos con conectividad limitada.  

%\begin{table}[H]
%    \centering
%    \begin{tabular}{l p{4cm} p{4cm} p{4cm}}
%    \toprule
%    \textbf{Criterio} & \textbf{Solución propuesta} & \textbf{Soluciones comerciales} & \textbf{Soluciones académicas} \\
%    \midrule
%    Costo de implementación & Bajo (hardware económico + software abierto) & Alto (licencias y servicios administrados) & Medio \\
%    Flexibilidad & Alta (protocolos estándar, código abierto) & Baja (plataformas propietarias) & Media \\
%    Escalabilidad & Alta (MQTT + API REST modular) & Alta & Media \\
%    Operación con conectividad intermitente & Soportada (colas FIFO y reintentos) & Parcial & Poco explorada \\
%    Adecuación a rutas argentinas & Específicamente adaptada & Genérica & Variable \\
%    \bottomrule
%    \end{tabular}
%    \caption{Comparación de la solución propuesta con alternativas comerciales y académicas.}
%    \label{tab:comparacion}
%\end{table}


%\chapter{Conclusiones}
\label{Chapter5}

En este capítulo se presentan las conclusiones generales del trabajo, junto con una reflexión sobre los resultados alcanzados y las posibles líneas de desarrollo futuro.  

%----------------------------------------------------------------------------------------
\section{Resultados y metas}
\label{sec:resultados_metas}

El sistema desarrollado permitió cumplir los objetivos propuestos, logra una solución integral para la detección y transmisión de eventos de tránsito en entornos con conectividad limitada.  
La arquitectura implementada demostró ser eficiente, modular y adaptable a distintos escenarios de despliegue, que mantiene la integridad de los datos y la estabilidad del flujo de comunicación entre nodos de campo, servidor y cliente web.

A continuación, se sintetizan los principales resultados y aportes del trabajo:

\begin{itemize}
  \item Se logró diseñar e implementar un sistema distribuido basado en protocolos abiertos (MQTT, REST y JSON), capaz de operar de forma confiable ante conectividad intermitente.  

  \item El firmware embebido en el ESP32-C3 validó su capacidad para procesar tramas RS-232, almacenar eventos temporalmente y asegurar su retransmisión una vez restablecida la conexión.  

  \item Se desarrolló una biblioteca específica para el manejo del módulo SIM800L, que permite ejecutar comandos AT y establecer la comunicación con el broker MQTT y que garantiza la publicación y suscripción de mensajes bajo condiciones variables de red.  
  \item El backend, desarrollado en Node.js con Express y MySQL, garantizó la persistencia de los datos, la trazabilidad de los eventos y la autenticación segura mediante tokens JWT.  
  
  \item El broker MQTT (Eclipse Mosquitto) permitió la comunicación asincrónica y la publicación confiable de mensajes entre dispositivos y servidor, con soporte para retención de mensajes.  
  
  \item La interfaz web, desarrollada en Ionic y Angular, proporcionó una visualización clara y funcional de los eventos de tránsito, junto con la posibilidad de emitir comandos y supervisar el estado de los nodos en tiempo real.  
  
  \item Las pruebas realizadas en laboratorio  demostraron la estabilidad del sistema y su capacidad de recuperación ante fallas o desconexiones.  
  
  \item La arquitectura modular permitió aislar componentes, facilitar el mantenimiento y posibilitar futuras extensiones sin comprometer la estabilidad del sistema base.
\end{itemize}

El cronograma original se mantuvo en líneas generales, con ajustes menores asociados a la disponibilidad de hardware y a la calibración de los tiempos de respuesta del módulo GPRS.  
Los riesgos identificados durante la planificación, vinculados principalmente a la inestabilidad del enlace móvil y a la gestión de colas locales, fueron mitigados mediante estrategias de reconexión automática y verificación de integridad de datos, que demostraron ser efectivas en las pruebas finales.


%----------------------------------------------------------------------------------------
\section{Trabajo futuro}
\label{sec:trabajo_futuro}

El trabajo realizado es un punto de partida para la evolución del sistema hacia una plataforma más completa, eficiente y adaptable a entornos operativos reales.  
Entre las líneas de continuidad propuestas se destacan las siguientes:

\begin{itemize}
  \item Incorporar un módulo de comunicación que integre nativamente el protocolo MQTT, en reemplazo del SIM800L. Esta modificación permitiría reducir la complejidad del firmware, aumentar la confiabilidad del enlace y mejorar los tiempos de transmisión.  

  \item Implementar mecanismos de actualización remota OTA para el firmware de los nodos de campo, con el fin de simplificar el mantenimiento y asegurar la uniformidad de versiones en despliegues múltiples.  
  
 \item Incorporar un módulo de análisis histórico y visualización avanzada que permita detectar patrones de tránsito, generar reportes automáticos y apoyar la toma de decisiones operativas.  
 
  \item Integrar servicios de geolocalización y mapas interactivos para representar la ubicación de los dispositivos y eventos en tiempo real.  
 
  \item Evaluar el desempeño del sistema en entornos de campo prolongados, con el propósito de obtener métricas de confiabilidad, consumo energético y latencia bajo condiciones reales de operación.  

  \item Desarrollar herramientas de monitoreo remoto y alertas automáticas, que notifiquen anomalías en los dispositivos o interrupciones de comunicación sin intervención manual.  

  \item Incorporar técnicas de inteligencia artificial para la detección de tránsito no registrado, lo que permitiría incrementar la precisión del sistema en escenarios con oclusión parcial o interferencias.  
  \item Migrar la comunicación móvil a redes 4G o 5G, con el fin de mejorar la disponibilidad, reducir la latencia y soportar mayor volumen de datos en despliegues masivos.  
\end{itemize}




%----------------------------------------------------------------------------------------
% Apéndices
%----------------------------------------------------------------------------------------

\appendix

% Incluir apéndices desde archivos separados si es necesario
%\include{Appendices/AppendixA}

%----------------------------------------------------------------------------------------
% Bibliografía
%----------------------------------------------------------------------------------------

\renewcommand{\bibname}{Bibliografía} % Para asegurarte de que el título sea correcto
\phantomsection % Necesario para que el enlace del marcador sea correcto
\nocite{*}
\printbibliography[heading=bibintoc, title={Bibliografía}]


\end{document}






